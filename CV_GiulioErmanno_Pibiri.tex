\documentclass[8pt, a4paper, sans]{moderncv}
\usepackage[utf8]{inputenc}

\moderncvstyle{casual}
\moderncvcolor{black}

\usepackage[left=2cm, bottom=4cm, right=2cm, top=4cm]{geometry}

\firstname{Giulio Ermanno}
\familyname{Pibiri}

\title{Curriculum Vitae et Studiorum}

\begin{document}

\makecvtitle

\section{Contact Information}
\cvitem{}{National Research Council of Italy (CNR)}
\cvitem{}{Institute of Science and Information Technologies ``A. Faedo'' (ISTI)}
\cvitem{}{Via G. Moruzzi 1, 56124 Pisa, Italy}
\cvitem{Email}{giulio.pibiri@di.unipi.it}
\cvitem{Email}{giulio.ermanno.pibiri@isti.cnr.it}
\cvitem{Personal page}{http://pages.di.unipi.it/pibiri}

\section{Personal Information}
\cvitem{Place of birth}{Florence, Italy}
\cvitem{Date of birth}{13 July 1990}

\section{Research Interests}
\cvitem{}{Indexing, Efficiency, Data Compression, Algorithm Engineering}

\section{Education}

\cventry{2015 -- 2018}{PhD in Computer Science}{}{}{}
{\begin{itemize}
\item University of Pisa, Pisa, Italy
\item Thesis: \emph{Space- and Time-Efficient Data Structures for Massive Datasets}.\\Defended on 08/03/2019.
\item Grade: Excellent.
\item Supervisor: Rossano Venturini (\url{http://pages.di.unipi.it/rossano})
\end{itemize}}

\cventry{2012 -- 2014}{Master Degree in Computer Science \& Networking}{}{}{}{
\begin{itemize}
\item University of Pisa and Scuola Superiore Sant'Anna, Pisa, Italy
\item Thesis: \emph{Dynamic Elias-Fano Encoding}.\\Defended on 06/03/2015.
\item Grade: 110/110 \emph{summa cum laude}.
\item Supervisor: Rossano Venturini (\url{http://pages.di.unipi.it/rossano})
\end{itemize}}

\cventry{2009 -- 2012}{Bachelor Degree in Computer Engineering}{}{}{}{
\begin{itemize}
\item University of Florence, Florence, Italy
\item Thesis: \emph{Quantum Computation \& Grover's Algorithm}.\\Defended on 09/10/2012.
\item Grade: 110/110 \emph{summa cum laude}.
\item Supervisor: Gabriele Vezzosi (\url{http://www.dma.unifi.it/~vezzosi})
\end{itemize}}

%\cventry{07/2012 -- 10/2012}{Quantum Mechanics and Quantum Computation on-line course}{University of Berkeley}{Berkeley, California}{}{}

\cventry{2004 -- 2009}{High School Diploma}{}{}{}{
\begin{itemize}
\item Liceo Scientifico Statale Guido Castelnuovo, Florence, Italy
\item Grade: 100/100.
\end{itemize}
}

\section{Professional Employment}
\cventry{11/2018 -- present}{Postdoctoral Researcher in Computer Science}{}{}{}{
\begin{itemize}
\item Institute of Science and Information Technologies ``A. Faedo'' (ISTI), National Research Council of Italy (CNR), Pisa, Italy
\item HPC-Lab headed by Raffaele Perego (\url{http://raffaele.isti.cnr.it})
\end{itemize}
}

\cventry{06/2017 -- present}{Software Developer at ISTI-CNR}{}{}{}{
\begin{itemize}
\item European project \emph{Big Data Grapes}, Indexing and Distributed Components, Pisa, Italy
\item European project \emph{Large-scale Indie Gaming Analytics}, Pisa, Italy
\end{itemize}
}

\cventry{11/2015 -- 11/2018}{PhD Student in Computer Science}{}{}{}{
\begin{itemize}
\item University of Pisa, Pisa, Italy
\item Thesis: \emph{Space- and Time-Efficient Data Structures for Massive Datasets}.
\item Supervisor: Rossano Venturini (\url{http://pages.di.unipi.it/rossano})
\end{itemize}}

\cventry{05/2018 -- 10/2018}{Visiting PhD Student}{}{}{}{
\begin{itemize}
\item The University of Melbourne, School of Computing and Information Systems, Melbourne, Australia
\item Supervisor: Alistair Moffat (\url{https://people.eng.unimelb.edu.au/ammoffat})
\end{itemize}
}

\cventry{04/2018 -- 05/2018}{Visiting PhD Student}{}{}{}{
\begin{itemize}
\item RIKEN Advanced Intelligence Project (AIP), Tokyo, Japan
\item Supervisor: Yasuo Tabei (\url{https://sites.google.com/site/yasuotabei})
\end{itemize}
}

\cventry{02/2016 -- present}{Teaching Assistant}{}{}{}{
\begin{itemize}
\item Algorithmics and Laboratory, Bachelor Degree in Computer Science, University of Pisa, Italy
\item Competitive Programming and Contests, Master Degree in Computer Science, University of Pisa, Italy
\end{itemize}}

\cventry{04/2015 -- 07/2015}{Software Engineering Intern at IBM}{}{}{}{Rome, Italy}

\section{Awards and Grants}

\cvitem{2017}{\emph{SIGIR Student Travel Grant} issued by ACM SIGIR.}{}{}{}
\cvitem{2015}{\emph{PhD Scholarship}
issued by the University of Pisa, Department of Computer Science.}{}{}{}
\cvitem{2015}{\emph{Master Degree Award: Best Performance a.y. 2013/2014}
issued by Scuola Superiore Sant'Anna.}{}{}{}
\cvitem{2015}{\emph{Best Master Thesis Award in Theoretical Computer Science}, issued by the Italian chapter of the European Association for Theoretical Computer Science (EATCS).}{}{}{}

\section{Publications}

\cvitem{2020}{Raffaele Perego, Giulio Ermanno Pibiri and Rossano Venturini, \textbf{Compressed Indexes for Fast Search of Semantic Data}. IEEE Transactions on Knowledge and Data Engineering (TKDE), pages 12.}

\cvitem{2019}{Giulio Ermanno Pibiri and Rossano Venturini, \textbf{Techniques for Inverted Index Compression}. CoRR, \url{http://arxiv.org/abs/1908.10598
}, pages 35.}

\cvitem{2019}{Giulio Ermanno Pibiri, \textbf{On Slicing Sorted Integer Sequences}. CoRR, \url{https://arxiv.org/abs/1907.01032}, pages 20.}

\cvitem{2019}{Giulio Ermanno Pibiri and Rossano Venturini, \textbf{On Optimally Partitioning Variable-Byte Codes}. IEEE Transactions on Knowledge and Data Engineering (TKDE), pages 12.}

\cvitem{2019}{Giulio Ermanno Pibiri and Rossano Venturini, \textbf{Handling Massive N-Gram Datasets Efficiently}. ACM Transactions on Information Systems (TOIS), pages 41.}

\cvitem{2019}{Giulio Ermanno Pibiri, Matthias Petri, Alistair Moffat, \textbf{Fast Dictionary-based Compression for Inverted Indexes}. ACM Conference on Web Search and Data Mining (WSDM), pages 9.}

\cvitem{2018}{Giulio Ermanno Pibiri and Rossano Venturini, \textbf{Variable-Byte Encoding is Now Space-Efficient Too}. CoRR, \url{https://arxiv.org/abs/1804.10949}, pages 14.}

\cvitem{2018}{Giulio Ermanno Pibiri and Rossano Venturini, \textbf{Inverted Index Compression}. Encyclopedia of Big Data Technologies, pages 8.}

\cvitem{2017}{Giulio Ermanno Pibiri and Rossano Venturini, \textbf{Efficient Data Structures for Massive \emph{N}-Gram Datasets}. ACM Conference on Research and Development in Information Retrieval (SIGIR), pages 10.}

\cvitem{2017}{Giulio Ermanno Pibiri and Rossano Venturini, \textbf{Dynamic Elias-Fano Representation}. Annual Symposium on Combinatorial Pattern Matching (CPM), pages 14.}

\cvitem{2017}{Giulio Ermanno Pibiri and Rossano Venturini, \textbf{Clustered Elias-Fano Indexes}. ACM Transactions on Information Systems (TOIS), volume 2, pages 33.}

\section{Software}

\cvitem{GitHub page}{https://github.com/jermp}

\cvitem{\href{https://github.com/jermp/essentials}{\textbf{essentials}}}{A \textsf{C++} library providing essential core utilities for data structure design and benchmarking.}

\cvitem{\href{https://github.com/jermp/2i_bench}{\textbf{2i\_bench}}}{The \textsf{C++} library used for the experiments in the paper \emph{Techniques for Inverted Index Compression}.}

\cvitem{\href{https://github.com/jermp/s_indexes}{\textbf{s\_indexes}}}{The \textsf{C++} library used for the experiments in the paper \emph{On Slicing Sorted Integer Sequences}.}

\cvitem{\href{https://github.com/jermp/rdf_indexes}{\textbf{rdf\_indexes}}}{The \textsf{C++} library used for the experiments in the paper \emph{Compressed Indexes for Fast Search of Semantic Data}.}

\cvitem{\href{https://github.com/jermp/dint}{\textbf{dint}}}{The \textsf{C++} library used for the experiments in the paper \emph{Fast Dictionary-based Compression for Inverted Indexes}.}

\cvitem{\href{https://github.com/jermp/opt_vbyte}{\textbf{opt\_vbyte}}}{The \textsf{C++} library used for the experiments in the paper \emph{On Optimally Partitioning Variable-Byte Codes}.}

\cvitem{\href{https://github.com/jermp/tongrams}{\textbf{tongrams}}}{The \textsf{C++} library implementing the compressed data structures and algorithms described in the papers \emph{Efficient Data Structures for Massive N-Gram Datasets} and \emph{Handling Massive N-Gram Datasets Efficiently}.}

\cvitem{\href{https://github.com/jermp/clustered_elias_fano_indexes}{\textbf{clustered\_indexes}}}{The \textsf{C++} library used for the experiments in the paper \emph{Clustered Elias-Fano Indexes}.}

\section{Talks}

\cvitem{07/06/2019}{\emph{Ordered Set Problems}. Seminar. ISTI-CNR, Pisa, Italy.}

\cvitem{08/03/2019}{\emph{Space- and Time-Efficient Data Structures}. PhD thesis defense. The University of Pisa, Pisa, Italy.}

\cvitem{12/02/2019}{\emph{Fast Dictionary-based Compression for Inverted Indexes}. WSDM conference presentation. Melbourne Exhibition Center, Melbourne, Australia.}

\cvitem{01/02/2019}{\emph{Indexing Compressed Data
for Fast Retrieval}. Talk. The University of Pisa, Pisa, Italy.}

\cvitem{15/11/2018}{\emph{Space- and Time-Efficient Data Structures}. PhD research results. The University of Pisa, Pisa, Italy.}

\cvitem{29/10/2018}{\emph{Effective Web Graph
Representations}. Seminar. The University of Pisa, Pisa, Italy.}

\cvitem{17/05/2018}{\emph{On Optimally Partitioning Variable-Byte Index Data}. Seminar. RMIT University, Melbourne, Australia.}

\cvitem{10/04/2018}{\emph{Elias-Fano Encoding: a powerful tool for data structure design}. Seminar. RIKEN AIP, Tokyo, Japan.}

\cvitem{10/10/2017}{\emph{Space- and Time-Efficient Data Structures}. PhD research results. The University of Pisa, Pisa, Italy.}

\cvitem{10/08/2017}{\emph{Efficient Data Structures for Massive N-Gram Datasets}. SIGIR conference presentation. Keio Plaza Hotel, Tokyo, Japan.}

\cvitem{06/07/2017}{\emph{Dynamic Elias-Fano Representation}. CPM conference presentation. University Library of Warsaw, Warsaw, Poland.}

\cvitem{06/06/2017}{\emph{Efficient Data Structures for Massive N-Gram Datasets}. IIR conference presentation. Universit\`{a} della Szizzera Italiana, Lugano, Switzerland.}

\cvitem{17/10/2016}{\emph{Space- and Time-Efficient Data Structures} PhD thesis proposal. The University of Pisa, Pisa, Italy.}

\cvitem{21/06/2016}{\emph{Elias-Fano Encoding: succinct representation of monotone integer sequences with search operations}. Seminar. The University of Pisa, Pisa, Italy.}

\section{Teaching Experience}

\cventry{02/2019 -- 06/2016}{Assistant for \emph{Algorithmics and Laboratory - Corso A, code 008AA}}{}{}{}{Bachelor Degree in Computer Science, University of Pisa, Italy}

\cventry{10/2018 -- 12/2018}{Assistant for \emph{Competitive Programming and Contests, code 645AA}}{}{}{}{Master Degree in Computer Science, University of Pisa, Italy}

\cventry{09/2017 -- 12/2017}{Assistant for \emph{Competitive Programming and Contests, code 645AA}}{}{}{}{Master Degree in Computer Science, University of Pisa, Italy}

\cventry{10/2016 -- 12/2016}{Teacher for \emph{Algorithmics and Laboratory - Corso di recupero, code 008AA}}{}{}{}{Bachelor Degree in Computer Science, University of Pisa, Italy}

\cventry{02/2016 -- 06/2016}{Assistant for \emph{Algorithmics and Laboratory - Corso A, code 008AA}}{}{}{}{Bachelor Degree in Computer Science, University of Pisa, Italy}

\section{Professional Activities}

\cvitem{2020}{Member of the Organizing Committee of the 28-th edition of the Annual European Symposium on Algorithms (ESA 2020).}

\cvitem{2020}{Member of the Program Committee of the 43-rd edition of the International ACM SIGIR Conference on Research and Development in Information Retrieval, (SIGIR 2020).}

\cvitem{2019}{Member of the Program Committee of the 42-nd edition of the International ACM SIGIR Conference on Research and Development in Information Retrieval, (SIGIR 2019).}

\cvitem{2019}{Member of the Organizing Committee of the 30-th edition of the International Symposium on Combinatorial Pattern Matching (CPM 2019).}

\cvitem{2018}{Member of the Program Committee of the 2-nd edition of the Workshop on Knowledge Graphs and Semantics for Text Retrieval and Analysis (KG4IR), in conjuction with ACM SIGIR 2018.}

\cvitem{2017}{Member of the Organizing Committee of the 24-th edition of the International Symposium on String Processing and Information Retrieval (SPIRE 2017).}

\cvitem{2016}{Student volunteer for the organization of the 39-th edition of the International ACM SIGIR Conference on Research and Development in Information Retrieval, (SIGIR 2016).}

%\section{Computer skills}
%\cvitem{OS}{Mac OS X, Linux.}
%\cvitem{Programming}{C++ and C (advanced); Java, Python.}
%\cvitem{Architecture}{Knowledge of modern CPUs architecture.}
%\cvitem{Typography}{\LaTeX, Pages.}
%\cvitem{Database}{MySQL, MongoDB.}

\section{Professional Memberships}
\cvitem{2019 -- present}{IEEE Italy Section.
Membership number: 95683551.}

\cvitem{2017 -- present}{Association for Computing Machinery (ACM SIGIR).
Membership number: 6216262.}

\section{Languages}
\cvitemwithcomment{Italian}{Native}{CEFR level: C2}
\cvitemwithcomment{English}{Fluent}{CEFR level: C1}
\cventry{2018}{TOEFL iBT in English}{}{}{}{100 (HIGH level)}
\cventry{2008}{First Certificate in English (Level B2)}{}{}{}{University of Cambridge, Cambridge, United Kingdom}

\section{Driving Licences}
\cvitem{01/10/2010}{Driving License of type B}
\cvitem{23/01/2010}{Driving License of type A}

\section{Web}
\cvitem{Personal}{http://pages.di.unipi.it/pibiri}
\cvitem{GitHub}{https://github.com/jermp}
\cvitem{LinkedIn}{https://www.linkedin.com/in/giulio-ermanno-pibiri-496ab3ab/}

%\section{Interests}
%\cvitem{Computer Science}{Efficient algorithms and data-structures for storage and fast retrieval of large integer/string data sets. Succinct and compressed data-structures.}
%\cvitem{Personal}{Painting miniatures, doing sport, music and movies.}

\end{document}