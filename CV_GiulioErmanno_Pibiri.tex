\documentclass[8pt, a4paper, sans]{moderncv}
\usepackage[utf8]{inputenc}

\moderncvstyle{casual}
\moderncvcolor{black}

\usepackage[scale=0.75, left=2cm, bottom=4cm, right=2cm, top=4cm]{geometry}

\firstname{Giulio Ermanno}
\familyname{Pibiri}

%\address{}{}
%\mobile{}

\email{giulio.pibiri@di.unipi.it}
\title{Curriculum Vitae et Studiorum}

%\photo[65pt][0pt]{}

\begin{document}

\makecvtitle

\section{Work experience}
\cventry{02/2019 -- present}{Teaching Assistant for Laboratory of Algorithms}{University of Pisa, Italy}{}{}{\url{http://didawiki.cli.di.unipi.it/doku.php/informatica/all-a/start}}

\cventry{10/2018 -- 12/2018}{Teaching Assistant for Competitive Programming}{University of Pisa, Italy}{}{}{\url{https://github.com/rossanoventurini/CompetitiveProgramming}}

\cventry{06/2017 -- 12/2017}{Software Developer for the European project Large-scale Indie Gaming Analytics (LIGA)}{ISTI-CNR, Pisa, Italy}{}{}{
Back-end design of a recommender system to suggest 3D models to users of the 3D-KUMO company.}

\cventry{09/2017 -- 12/2017}{Teaching Assistant for Competitive Programming}{University of Pisa, Italy}{}{}{\url{https://github.com/rossanoventurini/CompetitiveProgramming}}

\cventry{10/2016 -- 12/2016}{Teaching Assistant for Laboratory of Algorithms}{University of Pisa, Italy}{}{}{\url{http://didawiki.cli.di.unipi.it/doku.php/informatica/alr/start}}

\cventry{02/2016 -- 04/2016}{Teaching Assistant for Laboratory of Algorithms}{University of Pisa, Italy}{}{}{\url{http://didawiki.cli.di.unipi.it/doku.php/informatica/all-a/start}}

\cventry{04/2015 -- 07/2015}{Software Engineering Intern at IBM}{Rome, Italy}{}{}{Re-design of the IBM Customer Partnership Portal  with state-of-the-art tools, including MongoDB, NodeJS, ExpressJS, jQuery and Bootstrap. Proposed goal: an efficient portal implementation with a stunning and easy-to-use user interface.}

\section{Education}
\cventry{11/2018 - present}{Postdoctoral researcher in Computer Science}{Institute of Science and Information Technologies ``A. Faedo'', National Research Council (ISTI-CNR)}{Pisa, Italy}{}
{Research Area: Efficiency, Indexing, Algorithm Engineering, Data Compression}

\cventry{11/2015 - 10/2018}{PhD in Computer Science}{University of Pisa, Department of Computer Science}{Pisa, Italy}{}
{\begin{itemize}
\item Research Area: Efficiency, Indexing, Algorithm Engineering, Data Compression
\item Supervisor: Rossano Venturini (\url{http://pages.di.unipi.it/rossano})
\item Thesis: \emph{Space and Time-Efficient Data Structures for Massive Datasets}. Defended on 08/03/2019.
\end{itemize}}

\cventry{05/2018 - 10/2018}{Visiting PhD student at School of Computing and Information Systems}{}{Melbourne, Australia}{}{\url{https://cis.unimelb.edu.au/}}

\cventry{04/2018 - 05/2018}{Visiting PhD student at RIKEN AIP}{}{Tokyo, Japan}{}{\url{http://www.riken.jp/en/research/labs/aip/}}

\cventry{2012--2014}{Master Degree in Computer Science \& Networking}{University of Pisa and Scuola Superiore Sant'Anna di studi universitari e perfezionamento}{Pisa, Italy}{}{
\begin{itemize}
\item Supervisor: Rossano Venturini (\url{http://pages.di.unipi.it/rossano})
\item Thesis: \emph{Dynamic Elias-Fano Encoding}. Defended on 06/03/2015; \textbf{110/110 \emph{Summa Cum Laude}}.
\end{itemize}}

\cventry{2009--2012}{Bachelor Degree in Computer Engineering}{University of Florence}{Florence, Italy}{}{
\begin{itemize}
\item Supervisor: Gabriele Vezzosi (\url{http://www.dma.unifi.it/~vezzosi})
\item Thesis: \emph{Quantum Computation \& Grover's Algorithm}. Defended on 09/10/2012; \textbf{110/110 \emph{Summa Cum Laude}}.
\end{itemize}}

\cventry{07/2012 -- 10/2012}{Quantum Mechanics and Quantum Computation on-line course}{University of Berkeley}{Berkeley, California}{}{}

\cventry{2004--2009}{Diploma di Maturit\`{a} Scientifica}{Liceo Scientifico Statale Guido Castelnuovo}{Florence, Italy}{}{\textbf{100/100}}

\section{Publications}

\cvitem{2019}{Raffaele Perego, Giulio Ermanno Pibiri and Rossano Venturini, \textbf{Compressed Indexes for Fast Search of Semantic Data}. CoRR, \url{https://arxiv.org/abs/1904.07619}, pages 19.}

\cvitem{2019}{Giulio Ermanno Pibiri and Rossano Venturini, \textbf{On Optimally Partitioning Variable-Byte Codes}. IEEE Transactions on Knowledge and Data Engineering (TKDE), pages 12.}

\cvitem{2019}{Giulio Ermanno Pibiri and Rossano Venturini, \textbf{Handling Massive N-Gram Datasets Efficiently}. ACM Transactions on Information Systems (TOIS), pages 41.}

\cvitem{2019}{Giulio Ermanno Pibiri, Matthias Petri, Alistair Moffat, \textbf{Fast Dictionary-based Compression for Inverted Indexes}. ACM Conference on Web Search and Data Mining (WSDM), pages 9.}

\cvitem{2018}{Giulio Ermanno Pibiri and Rossano Venturini, \textbf{Variable-Byte Encoding is Now Space-Efficient Too}. CoRR, \url{https://arxiv.org/abs/1804.10949}, pages 14.}

\cvitem{2018}{Giulio Ermanno Pibiri and Rossano Venturini, \textbf{Inverted Index Compression}. Encyclopedia of Big Data Technologies, pages 8.}

\cvitem{2017}{Giulio Ermanno Pibiri and Rossano Venturini, \textbf{Efficient Data Structures for Massive \emph{N}-Gram Datasets}. ACM Conference on Research and Development in Information Retrieval (SIGIR), pages 10.}

\cvitem{2017}{Giulio Ermanno Pibiri and Rossano Venturini, \textbf{Dynamic Elias-Fano Representation}. Annual Symposium on Combinatorial Pattern Matching (CPM), pages 14.}

\cvitem{2017}{Giulio Ermanno Pibiri and Rossano Venturini, \textbf{Clustered Elias-Fano Indexes}. ACM Transactions on Information Systems (TOIS), volume 2, pages 33.}

\section{Code}

\cvitem{\href{https://github.com/jermp/rdf_indexes}{\textbf{rdf\_indexes}}}{The \textsf{C++} library used for the experiments in the paper \emph{Compressed Indexes for Fast Search of Semantic Data}.}

\cvitem{\href{https://github.com/jermp/dint}{\textbf{DINT}}}{The \textsf{C++} library used for the experiments in the paper \emph{Fast Dictionary-based Compression for Inverted Indexes}.}

\cvitem{\href{https://github.com/jermp/opt_vbyte}{\textbf{opt\_vbyte}}}{The \textsf{C++} library used for the experiments in the paper \emph{On Optimally Partitioning Variable-Byte Codes}.}

\cvitem{\href{https://github.com/jermp/tongrams}{\textbf{tongrams}}}{The \textsf{C++} library implementing the compressed data structures and algorithms described in the papers \emph{Efficient Data Structures for Massive N-Gram Datasets} and \emph{Handling Massive N-Gram Datasets Efficiently}.}

\cvitem{\href{https://github.com/jermp/clustered_elias_fano_indexes}{\textbf{clustered EF}}}{The \textsf{C++} library used for the experiments in the paper \emph{Clustered Elias-Fano Indexes}.}

\section{Awards and Grants}
\cvitem{2017}{\emph{SIGIR Student Travel Grant} issued by ACM SIGIR.}{}{}{}
\cvitem{2015}{\emph{Master Degree Award: Best Performance a.y. 2013/2014}
issued by Scuola Superiore Sant'Anna.}{}{}{}
\cvitem{2015}{\emph{Best Master Thesis Award in Theoretical Computer Science}, issued by the Italian chapter of the EATCS (European Association for Theoretical Computer Science). Invited to the Italian Conference on Theoretical Computer Science (ICTCS), Florence.}{}{}{}

\section{Conferences and talks}

\cvitem{2019}{WSDM Conference presentation entitled \emph{Fast Dictionary-based Compression for Inverted Indexes}. Melbourne Exhibition Center, Melbourne, Australia.}

\cvitem{2018}{Seminar talk entitled \emph{On Optimally Partitioning Variable-Byte Index Data}. RMIT University, Melbourne, Australia.}

\cvitem{2018}{Seminar talk entitled \emph{Elias-Fano Encoding: a powerful tool for data structure design}. Riken AIP, Tokyo, Japan.}

\cvitem{2017}{SIGIR Conference presentation entitled \emph{Efficient Data Structures for Massive N-Gram Datasets}. Keio Plaza Hotel, Tokyo, Japan.}

\cvitem{2017}{CPM Conference presentation entitled \emph{Dynamic Elias-Fano Representation}. University Library of Warsaw, Warsaw, Poland.}

\cvitem{2017}{IIR Conference presentation entitled \emph{Efficient Data Structures for Massive N-Gram Datasets}. Universit\`{a} della Szizzera Italiana, Lugano, Switzerland.}

\cvitem{2016}{Seminar talk entitled \emph{Elias-Fano Encoding: succinct representation of monotone integer sequences with search operations}. Department of Computer Science, Pisa, Italy.}

\section{Miscellanea}

\cvitem{2019}{Member of the Program Committee of the 42-nd edition of the International ACM SIGIR Conference on Research and Development in Information Retrieval, (SIGIR 2019).}

\cvitem{2018}{Member of the Organizing Committee of the 30-th edition of the International Symposium on Combinatorial Pattern Matching (CPM 2019).}

\cvitem{2018}{Member of the Program Committee of the 2-nd edition of the Workshop on Knowledge Graphs and Semantics for Text Retrieval and Analysis (KG4IR), in conjuction with ACM SIGIR 2018.}

\cvitem{2017}{Member of the Organizing Committee of the 24-th edition of the International Symposium on String Processing and Information Retrieval (SPIRE 2017).}

\cvitem{2016}{Student volunteer for the organization of the 39-th edition of the International ACM SIGIR Conference on Research and Development in Information Retrieval, (SIGIR 2016).}

\section{Computer skills}
\cvitem{OS}{Mac OS X, Linux.}
\cvitem{Programming}{C++ and C (advanced); Java, Python.}
\cvitem{Architecture}{Knowledge of modern CPUs architecture.}
\cvitem{Typography}{\LaTeX, Pages.}
\cvitem{Database}{MySQL, MongoDB.}

\section{Languages}
\cvitemwithcomment{Italian}{Native}{CEFR level: C2}
\cvitemwithcomment{English}{Fluent}{CEFR level: C1}

\cventry{2018}{TOEFL iBT}{100}{(HIGH level)}{}{}

\cventry{2008}{First Certificate in English, Council of Europe Level B2, Cambridge ESOL Level 1 Certificate in ESOL International}{University of Cambridge}{Cambridge, United Kingdom}{}{}

\section{Driving Licences}
\cvitem{01/10/2010}{Driving License of type B}
\cvitem{23/01/2010}{Driving License of type A}

\section{Web Sites}
\cvitem{Personal}{\url{http://pages.di.unipi.it/pibiri}}
\cvitem{Github}{\url{https://github.com/jermp}}
\cvitem{LinkedIn}{\url{https://it.linkedin.com/pub/giulio-ermanno-pibiri/ab/ab3/496}}

%\section{Interests}
%\cvitem{Computer Science}{Efficient algorithms and data-structures for storage and fast retrieval of large integer/string data sets. Succinct and compressed data-structures.}
%\cvitem{Personal}{Painting miniatures, doing sport, music and movies.}

\end{document}