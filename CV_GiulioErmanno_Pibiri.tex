\documentclass[a4paper, sans]{moderncv}
\usepackage[utf8]{inputenc}

\moderncvstyle{casual}
\moderncvcolor{black}

\usepackage[left=2cm, bottom=4cm, right=2cm, top=4cm]{geometry}



\newcommand{\todo}[1]{{\color{red}{\emph{#1}}}}


\firstname{Giulio Ermanno}
\familyname{Pibiri}

\title{Curriculum Vitae et Studiorum}

\begin{document}

% \textbf{Dichiarazione Sostitutiva di Certificazione (art. 46 DPR n. 445/2000)} \\
% \textbf{Dichiarazione Sostitutiva dell'Atto di Notorietà (art. 47 DPR n. 445/2000)} \\

% Il sottoscritto Giulio Ermanno Pibiri

% \begin{itemize}
%     \item nato a Bagno a Ripoli (FI) il 13/07/1990
%     \item attualmente residente a Prato (PO) in via Guizzelmi, n. 16, CAP 59100
%     \item reperibile al numero di telefono +39 340 9100050
% \end{itemize}
% \vspace{0.3cm}

% visto il D.P.R. 28 dicembre 2000, n. 445 concernente ``T.U. delle disposizioni
%     legislative e regolamentari in materia di documentazione amministrativa'' e successive
%     modifiche ed integrazioni;
% \vspace{0.3cm}

% vista la Legge 12 novembre 2011, n. 183 ed in particolare l'art. 15 concernente
%     le nuove disposizioni in materia di certificati e dichiarazioni sostituitive;
% \vspace{0.3cm}

% consapevole che, ai sensi dell'art. 76 del DPR 445/2000, le dichiarazioni mendaci, la falsità
% negli atti e l'uso di atti falsi sono punite ai sensi del Codice penale e delle leggi speciali vigenti
% in materia, dichiara sotto la propria responsabilità:
% che quanto dichiarato nel seguente \emph{Curriculum Vitae et Studiorum}
% comprensivo delle informazioni sulla produzione scientifica corrisponde a verità.




% \vspace{6cm}




\makecvtitle

\section{Contact Information}
\cvitem{}{National Research Council of Italy (CNR)}
\cvitem{}{Institute of Science and Information Technologies ``A. Faedo'' (ISTI)}
\cvitem{}{Via G. Moruzzi 1, 56124 Pisa, Italy}
\cvitem{Email}{giulio.pibiri@di.unipi.it}
\cvitem{Email}{giulio.ermanno.pibiri@isti.cnr.it}
\cvitem{Personal page}{\url{http://pages.di.unipi.it/pibiri}}
\cvitem{GitHub profile}{\url{https://github.com/jermp}}

\section{Personal Information}
\cvitem{Place of birth}{Bagno a Ripoli (Florence), Italy}
\cvitem{Date of birth}{13 July 1990}

\section{Research Interests}

\cvitem{Keywords}{Data Structures, Data Compression, Indexing, Efficiency}

\cvitem{Short description}{The research activity focuses on devising compressed data structures
and algorithms to index and search large quantities of data.
The proposed solutions are available as research papers
and optimized software libraries (written in C++).}

\cvitem{Studied problems}{
Succinct and dynamic ordered sets of integers,
inverted index compression,
indexing and estimation of language models,
indexing of semantic relations,
bitmap compression,
query auto-completion,
rank/select queries,
prefix-sums,
minimal perfect hashing.
}

\section{Education}

\cventry{01/11/2015 -- 31/10/2018}{PhD in Computer Science (INF/01)}{}{}{}
{\begin{itemize}
\item University of Pisa, Pisa, Italy
\item Thesis: \emph{Space- and Time-Efficient Data Structures for Massive Datasets}
\newline
(Defended on 08/03/2019)
\item Grade: Excellent
\item Supervisor: Rossano Venturini (\url{https://rossanoventurini.github.io})
\end{itemize}}

\cventry{2012 -- 2014}{Master Degree in Computer Science \& Networking (class LM18)}{}{}{}{
\begin{itemize}
\item University of Pisa and Scuola Superiore Sant'Anna, Pisa, Italy
\item Thesis: \emph{Dynamic Elias-Fano Encoding}
\newline
(Defended on 06/03/2015)
\item Grade: 110/110 \emph{summa cum laude}
\item Supervisor: Rossano Venturini (\url{https://rossanoventurini.github.io})
\end{itemize}}

\cventry{2009 -- 2012}{Bachelor Degree in Computer Engineering (class L08)}{}{}{}{
\begin{itemize}
\item University of Florence, Florence, Italy
\item Thesis: \emph{Quantum Computation \& Grover's Algorithm}
\newline
(Defended on 09/10/2012)
\item Grade: 110/110 \emph{summa cum laude}
\item Supervisor: Gabriele Vezzosi (\url{http://www.dma.unifi.it/~vezzosi})
\end{itemize}}

\cventry{2004 -- 2009}{High School Diploma}{}{}{}{
\begin{itemize}
\item Liceo Scientifico Statale Guido Castelnuovo, Florence, Italy
\item Grade: 100/100.
\end{itemize}
}

\section{Professional Employment}

\cventry{15/03/2021 -- present}{Postdoctoral Research Fellow in Computer Science}{}{}{}{
\begin{itemize}
\item Institute of Science and Information Technologies ``A. Faedo'' (ISTI), National Research Council of Italy (CNR), Pisa, Italy
\item Project: European project ACCORDION with theme ``Tecniche algoritmiche per compressione, indicizzazione e ricerca di grandi quantità di dati e progettazione di relative librerie software open source'' (Protocollo n. 0000901/2021, 09/03/2021, ISTI 004/2021 - PI).
\item Role: Principal Investigator
\end{itemize}
}

\cventry{01/11/2018 -- 28/02/2021}{Postdoctoral Research Fellow in Computer Science}{}{}{}{
\begin{itemize}
\item Institute of Science and Information Technologies ``A. Faedo'' (ISTI), National Research Council of Italy (CNR), Pisa, Italy
\item Project: European project BIGDATAGRAPES with theme ``Compressione, indicizzazione e ricerca su grandi collezioni di dati semantici'' (Protocollo n. 0003847, 24/10/2018, ISTI 014/2018 - PI).
\item Role: Principal Investigator
\item Description: The research activity conducted for this project focused on the design of time and space efficient indexing data structures for structured and unstructured data such as RDF graphs and text documents, including compression techniques for Big data management that support a broad range of analytical queries over arbitrary data dimensions.
In particular, it resulted in the development of
\begin{itemize}
    \item a novel compressor for inverted indexes;
    \item a novel compressed index for RDF data.
    % \item a novel compressor for time series.
\end{itemize}
Both results have been published in IEEE Transactions on Knowledge and Data Engineering (TKDE),
as the papers
\begin{itemize}
\item ``On Optimally Patitioning Variable-Byte Codes''
\item ``Compressed Indexes for Fast Search of Semantic Data''
\end{itemize}
with the corresponding C++ libraries available on GitHub at
\begin{itemize}
\item \url{https://github.com/jermp/opt_vbyte}
\item \url{https://github.com/jermp/rdf_indexes}
\end{itemize}
\item During this period, I also kept doing my own independent research on algorithms and data structures.
The studied problems involved:
inverted indexes (CSUR 2020),
prefix-sums (SPE 2020),
bitmap compression (DCC 2021),
query auto-completion (SIGIR 2020),
rank and select indexes for bitmaps (INFOSYS 2021),
and minimal perfect hashing (SIGIR 2021).
All works have been published in top-tier conferences/journals.
The corresponding software libraries are available from my GitHub page.
\end{itemize}
}

\cventry{01/06/2017 -- 31/10/2018}{Software Developer}{}{}{}{
\begin{itemize}
\item Institute of Science and Information Technologies ``A. Faedo'' (ISTI), National Research Council of Italy (CNR), Pisa, Italy
\item Project: European project \emph{Large-scale Indie Gaming Analytics} (LIGA).
\item Role: Principal Developer
\item Description: The LIGA project aimed at designing and developing a proof-of-concept platform, customized for the 3D-KUMO use case (\url{https://www.3dkumo.com}), to analyze the huge volume volume of data generated by the users of indie games (i.e., players) on web portals and social networks.
% The purpose of the LIGA platform is to extend the engagement of indie games’ users beyond the gaming experience to the collection and sharing of game-related content, mainly related to 3D printing of game characters.
% To this end, the LIGA platform consists in three main components:
% (1) A High Performance Big Data Analysis (HPC-BDA) tool for the near-real time fetching and representation of online social network data related to indie games, including trends, players sentiment and network structure analysis;
% (2) A Smart Retail Recommendation Engine (REC-ENG) service integrated with the KUMO platform to suggests 3D models to customers, based on the information provided by the HPC-BDA tool;
% (3) A 3D Content Analytics Service and Dashboard (3D-DASH) service to provide content providers with insights and analysis on why contents are consumed and ``liked'', based on the information provided by the HPC-BDA tool.
\end{itemize}
}

\cventry{01/11/2015 -- 31/10/2018}{PhD Student in Computer Science}{}{}{}{
\begin{itemize}
\item University of Pisa, Pisa, Italy
\item Thesis: \emph{Space- and Time-Efficient Data Structures for Massive Datasets}
\item Supervisor: Rossano Venturini (\url{https://rossanoventurini.github.io})
\item Worked on inverted indexing, compressed language models, and tries.
The thesis is based on the following publications.
\begin{itemize}
    \item ``On Optimally Partitioning Variable-Byte Codes'' (TKDE 2019) %, C++ code at \url{https://github.com/jermp/opt_vbyte};
    \item ``Handling Massive N-Gram Datasets Efficiently'' (TOIS 2019) %, C++ code at \url{https://github.com/jermp/tongrams};
    \item ``Fast Dictionary-based Compression for Inverted Indexes'' (WSDM 2019) %, C++ code at \url{https://github.com/jermp/dint};
    \item ``Inverted Index Compression'' (EBDT 2018)
    \item ``Efficient Data Structures for Massive N-Gram Datasets'' (SIGIR 2017) %, C++ code at \url{https://github.com/jermp/tongrams};
    \item ``Dynamic Elias-Fano Representation'' (CPM 2017)
    \item ``Clustered Elias-Fano Indexes'' (TOIS 2017) %, C++ code at \url{https://github.com/jermp/clustered_elias_fano_indexes}.
\end{itemize}
\end{itemize}}

\cventry{01/05/2018 -- 01/10/2018}{Visiting PhD Student}{}{}{}{
\begin{itemize}
\item The University of Melbourne, School of Computing and Information Systems, Melbourne, Australia
\item Supervisor: Alistair Moffat (\url{https://people.eng.unimelb.edu.au/ammoffat})
\item Worked on fast dictionary-based decoding of compressed inverted index data, which resulted in the
following publication: ``Fast Dictionary-based Compression for Inverted Indexes'' (WSDM 2019). %, C++ code at \url{https://github.com/jermp/dint}.
\end{itemize}
}

\cventry{01/04/2018 -- 30/04/2018}{Visiting PhD Student}{}{}{}{
\begin{itemize}
\item RIKEN Advanced Intelligence Project (AIP), Tokyo, Japan
\item Supervisor: Yasuo Tabei (\url{https://sites.google.com/site/yasuotabei})
\item Worked on various problems, such as, string similarity search,
trie indexing, rank/select indexes, and sparse matrix multiplication.
Reference publication: ``Rank/Select Queries over Mutable Bitmaps'' (INFOSYS 2021).
\end{itemize}
}

%\cventry{02/2016 -- present}{Teaching Assistant}{}{}{}{
%\begin{itemize}
%\item Algorithmics and Laboratory, Bachelor Degree in Computer Science, University of Pisa, Italy
%\item Competitive Programming and Contests, Master Degree in Computer Science, University of Pisa, Italy
%\end{itemize}}

\cventry{01/04/2015 -- 01/07/2015}{Software Engineer Intern at IBM}{}{}{}{
\begin{itemize}
\item Rome, Italy
\item Supervisor: Alessio Fioravanti
\item Worked on the design of the IBM Customer Partnership (Web) Portal for the management of IBM customers and projects.
\end{itemize}
}



\section{Teaching Experience}

\cvitem{02/2020 -- 06/2020}{Teacher for \emph{Algorithmics and Laboratory - Corso B, code 008AA, 3 CFU}, Bachelor Degree in Computer Science, University of Pisa, Italy}
% \cvitem{}{\url{http://didawiki.cli.di.unipi.it/doku.php/informatica/all-b/start}}

\cvitem{02/2019 -- 06/2019}{Assistant for \emph{Algorithmics and Laboratory - Corso A, code 008AA, 3 CFU}, Bachelor Degree in Computer Science, University of Pisa, Italy}
% \cvitem{}{\url{http://didawiki.cli.di.unipi.it/doku.php/informatica/all-b/start}}

\cvitem{09/2018 -- 12/2018}{Assistant for \emph{Competitive Programming and Contests, code 645AA, 6 hours}, Master Degree in Computer Science, University of Pisa, Italy}
% \cvitem{}{\url{https://github.com/rossanoventurini/CompetitiveProgramming/tree/47111d935e5de6755890dd2951742bdf4857ce3b}}

\cvitem{09/2017 -- 12/2017}{Assistant for \emph{Competitive Programming and Contests, code 645AA, 6 hours}, Master Degree in Computer Science, University of Pisa, Italy}
% \cvitem{}{\url{https://github.com/rossanoventurini/CompetitiveProgramming/tree/9583aa75dc0e2e87c8b0a167d7978360ad9595f5}}

\cvitem{09/2016 -- 12/2016}{Teacher for \emph{Algorithmics and Laboratory - Corso di recupero, code 008AA, 3 CFU}, Bachelor Degree in Computer Science, University of Pisa, Italy}
% \cvitem{}{\url{http://didawiki.cli.di.unipi.it/doku.php/informatica/alr/start}}

\cvitem{02/2016 -- 06/2016}{Assistant for \emph{Algorithmics and Laboratory - Corso A, code 008AA, 3 CFU}, Bachelor Degree in Computer Science, University of Pisa, Italy}
% \cvitem{}{\url{http://didawiki.cli.di.unipi.it/doku.php/informatica/all-a/all16/start}}



\section{Awards and Grants}

\cvitem{2020}{\emph{Young Researcher Award} issued by ISTI-CNR.}
% \cvitem{}{\url{https://www.isti.cnr.it/en/research/awards/young-researcher-award}}

\cvitem{2017}{\emph{SIGIR Student Travel Grant} issued by ACM SIGIR.}
% \cvitem{}{\url{https://sigir.org/general-information/travel-grants}}

\cvitem{2015}{\emph{PhD Scholarship}
issued by the University of Pisa, Department of Computer Science.}
% \cvitem{}{\url{https://dottorato.di.unipi.it/phd-programme/people/alumni}}

\cvitem{2015}{\emph{Master Degree Award: Best Performance a.y. 2013/2014}
issued by Scuola Superiore Sant'Anna.}
% \cvitem{}{\href{https://www.santannapisa.it/it/news/merito-nuovi-riconoscimenti-agli-studenti-della-laurea-magistrale-informatica-e-networking}{\texttt{https://www.santannapisa.it/it/news/merito}}}

\cvitem{2015}{\emph{Best Master Thesis Award in Theoretical Computer Science}, issued by the Italian chapter of the European Association for Theoretical Computer Science (EATCS).}
% \cvitem{}{\url{https://www.eatcs.org/index.php/italian-chapter}}

\section{Publications}

\cvitem{}{\textbf{Average number of co-authors per publication: 20/18 = 1.111}
\newline
(PhD Thesis excluded.)}

\cvitem{2021}{Giulio Ermanno Pibiri and Roberto Trani, \emph{Parallel and External-Memory Construction of Minimal Perfect Hash Functions with PTHash}. CoRR, \url{https://arxiv.org/abs/2106.02350}, pages 12.}

\cvitem{2021}{Giulio Ermanno Pibiri and Roberto Trani, \emph{PTHash: Revisiting FCH Minimal Perfect Hashing}. ACM Conference on Research and Development in Information Retrieval (SIGIR), pages 10.
\newline
GGS Rating: \textbf{A++}}

\cvitem{2021}{Giulio Ermanno Pibiri and Shunsuke Kanda, \emph{Rank/Select Queries over Mutable Bitmaps}. Information Systems (INFOSYS), pages 21.
\newline
Scimago Rating: \textbf{Q2}}

\cvitem{2021}{Raffaele Perego, Giulio Ermanno Pibiri and Rossano Venturini, \emph{Compressed Indexes for Fast Search of Semantic Data}. IEEE International Conference on Data Engineering (ICDE), pages 2.
\newline
GGS Rating: \textbf{A++}}

\cvitem{2021}{Giulio Ermanno Pibiri, \emph{Fast and Compact Set Intersection through Recursive Universe Partitioning}. IEEE Data Compression Conference (DCC), pages 10.
\newline
GGS Rating: \textbf{A-}}

\cvitem{2020}{Giulio Ermanno Pibiri and Rossano Venturini, \emph{Techniques for Inverted Index Compression}. ACM Computing Surveys (CSUR), pages 36.
\newline
Scimago Rating: \textbf{Q1}}

\cvitem{2020}{Giulio Ermanno Pibiri and Rossano Venturini, \emph{Practical Trade-Offs for the Prefix-Sum Problem}. Software: Practice and Experience (SPE), pages 29.
\newline
Scimago Rating: \textbf{Q2}}

\cvitem{2020}{Simon Gog, Giulio Ermanno Pibiri and Rossano Venturini, \emph{Efficient and Effective Query Auto-Completion}. ACM Conference on Research and Development in Information Retrieval (SIGIR), pages 10.
\newline
GGS Rating: \textbf{A++}}

\cvitem{2020}{Giulio Ermanno Pibiri and Rossano Venturini, \emph{Succinct Dynamic Ordered Sets with Random Access}. CoRR, \url{https://arxiv.org/abs/2003.11835}, pages 15.}

\cvitem{2020}{Raffaele Perego, Giulio Ermanno Pibiri and Rossano Venturini, \emph{Compressed Indexes for Fast Search of Semantic Data}. IEEE Transactions on Knowledge and Data Engineering (TKDE), pages 12.
\newline
Scimago Rating: \textbf{Q1}}

\cvitem{2019}{Giulio Ermanno Pibiri. \emph{On Implementing the Binary Interpolative Coding Algorithm}. Tech Report, 8 pages.}

\cvitem{2019}{Giulio Ermanno Pibiri. \emph{Space- and Time-Efficient Data Structures for Massive Datasets}. Ph.D. Thesis, University of Pisa, 210 pages.}

\cvitem{2019}{Giulio Ermanno Pibiri and Rossano Venturini, \emph{On Optimally Partitioning Variable-Byte Codes}. IEEE Transactions on Knowledge and Data Engineering (TKDE), pages 12.
\newline
Scimago Rating: \textbf{Q1}}

\cvitem{2019}{Giulio Ermanno Pibiri and Rossano Venturini, \emph{Handling Massive N-Gram Datasets Efficiently}. ACM Transactions on Information Systems (TOIS), pages 41.
\newline
Scimago Rating: \textbf{Q1}}

\cvitem{2019}{Giulio Ermanno Pibiri, Matthias Petri, Alistair Moffat, \emph{Fast Dictionary-based Compression for Inverted Indexes}. ACM Conference on Web Search and Data Mining (WSDM), pages 9.
\newline
GGS Rating: \textbf{A+}}

% \cvitem{2018}{Giulio Ermanno Pibiri and Rossano Venturini, \emph{Variable-Byte Encoding is Now Space-Efficient Too}. CoRR, \url{https://arxiv.org/abs/1804.10949}, pages 14.}

\cvitem{2018}{Giulio Ermanno Pibiri and Rossano Venturini, \emph{Inverted Index Compression}. Encyclopedia of Big Data Technologies (EBDT), pages 8.}

\cvitem{2017}{Giulio Ermanno Pibiri and Rossano Venturini, \emph{Efficient Data Structures for Massive N-Gram Datasets}. ACM Conference on Research and Development in Information Retrieval (SIGIR), pages 10.
\newline
GGS Rating: \textbf{A++}}

\cvitem{2017}{Giulio Ermanno Pibiri and Rossano Venturini, \emph{Dynamic Elias-Fano Representation}. Annual Symposium on Combinatorial Pattern Matching (CPM), pages 14.
\newline
GGS Rating: B}

\cvitem{2017}{Giulio Ermanno Pibiri and Rossano Venturini, \emph{Clustered Elias-Fano Indexes}. ACM Transactions on Information Systems (TOIS), volume 2, pages 33.
\newline
Scimago Rating: \textbf{Q1}}


\newpage

\section{Software}

\cvitem{GitHub profile}{\url{https://github.com/jermp}}

\bigskip
\subsection{\textbf{Data Structures}}

\cvitem{}{At my GitHub profile you can find efficient C++ implementations of the following data structures (see also related publications):
\begin{itemize}
\item Inverted Indexes (TOIS 2017, TKDE 2019, WSDM 2019, SIGIR 2020, CSUR 2020)
\item Tries (SIGIR 2017, TOIS 2019, TKDE 2020)
\item Compressed Bitmaps (DCC 2021)
\item Mutable Bitmaps with Rank/Select (INFOSYS 2021)
\item Segment-Trees and Fenwick-Trees (SPE 2020)
\item Minimal Perfect Hash Functions (SIGIR 2021)
\end{itemize}
A more detailed list follows below.
}

\cvitem{\href{https://github.com/jermp/pthash}{\textbf{pthash}}}{
PTHash: Fast and compact minimal perfect hash functions.
\newline
Reference publications: SIGIR 2021, arXiv 2106.02350 2021.
% The \textsf{C++} library used for the experiments in the papers \emph{PTHash: Revisiting FCH Minimal Perfect Hashing} and \emph{Parallel and External-Memory Construction of Minimal Perfect Hash Functions with PTHash}.
}

\cvitem{\href{https://github.com/jermp/mutable_rank_select}{\textbf{rank\_select}}}{
Mutable bitmaps with support for Rank and Select queries.
\newline
Reference publication: INFOSYS 2021.
% The \textsf{C++} library used for the experiments in the paper \emph{Rank/Select Queries over Mutable Bitmaps}.
}

\cvitem{\href{https://github.com/jermp/psds}{\textbf{psds}}}{
A range of tree-shaped data structures for maintaining prefix-sums, including:
\begin{itemize}
\item binary Segment-Tree (top-down and bottom-up),
\item b-ary Segment-Tree,
\item Fenwick-Tree,
\item b-ary Fenwick-Tree,
\item blocked Fenwick-Tree,
\item truncated Fenwick-Tree.
\end{itemize}
Reference publication: SPE 2020.
% The \textsf{C++} library used for the experiments in the paper \emph{Practical Trade-Offs for the Prefix-Sum Problem}.
}

\cvitem{\href{https://github.com/jermp/autocomplete}{\textbf{autocomplete}}}{Efficienct and effective
autocompletion framework, based on forward/inverted indexes, succinct RMQ, and string dictionaries (Front-Coding and tries).
\newline
Reference publication: SIGIR 2020.
% The \textsf{C++} library used for the experiments in the paper \emph{Efficient and Effective Query Auto-Completion}.
}

\cvitem{\href{https://github.com/jermp/2i_bench}{\textbf{2i\_bench}}}{
A benchmarking suite for inverted index data structures, featuring the following compressors:
\begin{itemize}
\item Elias-Fano and partitioned Elias-Fano,
\item Opt-PFor-Delta,
\item Binary Interpolative,
\item QMX,
\item Simple family,
\item Variable-Byte family, including Opt-VByte,
\item Gamma, Delta, Rice, Zeta,
\item DINT.
\end{itemize}
Reference publication: CSUR 2020.
% The \textsf{C++} library used for the experiments in the paper \emph{Techniques for Inverted Index Compression}.
}

\cvitem{\href{https://github.com/jermp/interpolative_coding}{\textbf{interp}}}{
An efficient implementation of the Binary Interpolative Coding
algorithm.
}


\cvitem{\href{https://github.com/jermp/s_indexes}{\textbf{s\_indexes}}}{
Compressed bitmap indexes that support fast intersection and union.
\newline
Reference publication: DCC 2021.
% The \textsf{C++} library used for the experiments in the paper \emph{Fast and Compact Set Intersection thorough Recursive Universe Partitioning}.
}

\cvitem{\href{https://github.com/jermp/rdf_indexes}{\textbf{rdf\_indexes}}}{
Trie-based indexes for semantic data like RDF triples.
\newline
Reference publication: TKDE 2020.
% The \textsf{C++} library used for the experiments in the paper \emph{Compressed Indexes for Fast Search of Semantic Data}.
}

\cvitem{\href{https://github.com/jermp/dint}{\textbf{dint}}}{
DINT: fast and compact dictionary-based decoder for inverted lists.
\newline
Reference publication: WSDM 2019.
% The \textsf{C++} library used for the experiments in the paper \emph{Fast Dictionary-based Compression for Inverted Indexes}.
}

\cvitem{\href{https://github.com/jermp/opt_vbyte}{\textbf{opt\_vbyte}}}{
Optimal partitioning of inverted lists compressed using binary vectors
and point-wise encoders, like Variable-Byte.
\newline
Reference publication: TKDE 2019.
% The \textsf{C++} library used for the experiments in the paper \emph{On Optimally Partitioning Variable-Byte Codes}.
}

\cvitem{\href{https://github.com/jermp/tongrams}{\textbf{tongrams}}}{
Fast language model queries and estimation in compressed space.
\newline
Reference publications: SIGIR 2017, TOIS 2019.
% The \textsf{C++} library implementing the compressed data structures and algorithms described in the papers \emph{Efficient Data Structures for Massive N-Gram Datasets} and \emph{Handling Massive N-Gram Datasets Efficiently}.
}

\cvitem{\href{https://github.com/jermp/clustered_elias_fano_indexes}{\textbf{clustered\_indexes}}}{
Clustered Elias-Fano inverted indexes.
\newline
Reference publication: TOIS 2017.
% The \textsf{C++} library used for the experiments in the paper \emph{Clustered Elias-Fano Indexes}.
}

\bigskip
\subsection{\textbf{Miscellanea}}

\cvitem{\href{https://github.com/jermp/essentials}{\textbf{essentials}}}{A \textsf{C++} library providing essential core utilities for data structure design and benchmarking.
More precisely:
\begin{itemize}
\item benchmarking facilities, including: messages displaying local time, configurable timer class, function to prevent code elision by compiler, simple creation and printing of json documents;
\item functions to serialize-to and load-from disk data structures,
\item functions to compute the numbr of bytes consumed by data structures,
\item support for creating, removing, and iterate inside directories,
\item transparent support for contiguous memory allocation.
\end{itemize}
}

\cvitem{\href{https://github.com/jermp/cmd_line_parser}{\textbf{cmd\_line}}}{
Command line parser for C++17. It offers all handy features in just 150 lines of code.
}

\cvitem{\href{https://github.com/jermp/mm_file}{\textbf{mm\_file}}}{
A self-contained, header-only, implementation of memory-mapped files in C++ for both reading and writing.
}


\section{Talks}

\cvitem{04/2020}{\emph{Compressed Indexes for Fast Search of Semantic Data}. ICDE conference presentation. Virtual event.}

\cvitem{03/2020}{\emph{Fast and Compact Set Intersection through Recursive Universe Partitioning}. DCC conference presentation. Virtual event.}

\cvitem{04/03/2020}{\emph{Efficiency for Real-World Applications} Seminar. ISTI-CNR. Virtual event.}

\cvitem{27/07/2020}{\emph{Efficient and Effective Query Auto-Completion}. SIGIR conference presentation. Virtual event.}

\cvitem{17/09/2019}{\emph{Compressed Indexes for Fast Search of Semantic Data}. IIR conference presentation. Department of Information Engineering, Padova, Italy.}

\cvitem{07/06/2019}{\emph{Ordered Set Problems}. Seminar. ISTI-CNR, Pisa, Italy.}

\cvitem{08/03/2019}{\emph{Space- and Time-Efficient Data Structures}. PhD thesis defense. The University of Pisa, Pisa, Italy.}

\cvitem{12/02/2019}{\emph{Fast Dictionary-based Compression for Inverted Indexes}. WSDM conference presentation. Melbourne Exhibition Center, Melbourne, Australia.}

\cvitem{01/02/2019}{\emph{Indexing Compressed Data
for Fast Retrieval}. Talk. The University of Pisa, Pisa, Italy.}

\cvitem{15/11/2018}{\emph{Space- and Time-Efficient Data Structures}. PhD research results. The University of Pisa, Pisa, Italy.}

\cvitem{29/10/2018}{\emph{Effective Web Graph
Representations}. Seminar. The University of Pisa, Pisa, Italy.}

\cvitem{17/05/2018}{\emph{On Optimally Partitioning Variable-Byte Index Data}. Seminar. RMIT University, Melbourne, Australia.}

\cvitem{10/04/2018}{\emph{Elias-Fano Encoding: a powerful tool for data structure design}. Seminar. RIKEN AIP, Tokyo, Japan.}

\cvitem{10/10/2017}{\emph{Space- and Time-Efficient Data Structures}. PhD research results. The University of Pisa, Pisa, Italy.}

\cvitem{10/08/2017}{\emph{Efficient Data Structures for Massive N-Gram Datasets}. SIGIR conference presentation. Keio Plaza Hotel, Tokyo, Japan.}

\cvitem{06/07/2017}{\emph{Dynamic Elias-Fano Representation}. CPM conference presentation. University Library of Warsaw, Warsaw, Poland.}

\cvitem{06/06/2017}{\emph{Efficient Data Structures for Massive N-Gram Datasets}. IIR conference presentation. Universit\`{a} della Szizzera Italiana, Lugano, Switzerland.}

\cvitem{17/10/2016}{\emph{Space- and Time-Efficient Data Structures} PhD thesis proposal. The University of Pisa, Pisa, Italy.}

\cvitem{21/06/2016}{\emph{Elias-Fano Encoding: succinct representation of monotone integer sequences with search operations}. Seminar. The University of Pisa, Pisa, Italy.}


\section{Professional Activities}

\cvitem{2021}{Member of the Program Committee of the 30-th edition of the International ACM Conference on Information and Knowledge Management (CIKM 2021).}

\cvitem{2021}{Member of the Program Committee of the 44-th edition of the International ACM SIGIR Conference on Research and Development in Information Retrieval (SIGIR 2021).}

\cvitem{2021}{Member of the Program Committee of the 43-rd European Conference on Information Retrieval (ECIR 2021).}

\cvitem{2020}{Member of the Program Committee of the 14-th International ACM Conference on Web Search and Data Mining (WSDM 2021).}

\cvitem{2020}{Member of the Program Committee of the 29-th edition of the International ACM Conference on Information and Knowledge Management (CIKM 2020).}

\cvitem{2020}{Member of the Organizing Committee of the 28-th edition of the Annual European Symposium on Algorithms (ESA 2020).}

\cvitem{2020}{Member of the Program Committee of the 43-rd edition of the International ACM SIGIR Conference on Research and Development in Information Retrieval (SIGIR 2020).}

\cvitem{2019}{Member of the Program Committee of the 42-nd edition of the International ACM SIGIR Conference on Research and Development in Information Retrieval (SIGIR 2019).}

\cvitem{2019}{Member of the Organizing Committee of the 30-th edition of the International Symposium on Combinatorial Pattern Matching (CPM 2019).}

\cvitem{2018}{Member of the Program Committee of the 2-nd edition of the Workshop on Knowledge Graphs and Semantics for Text Retrieval and Analysis (KG4IR), in conjuction with ACM SIGIR 2018.}

\cvitem{2017}{Member of the Organizing Committee of the 24-th edition of the International Symposium on String Processing and Information Retrieval (SPIRE 2017).}

\cvitem{2016}{Student volunteer for the organization of the 39-th edition of the International ACM SIGIR Conference on Research and Development in Information Retrieval (SIGIR 2016).}

\cvitem{2016 -- present}{Anonymous reviewer for the following conferences/journals:
SIGIR, WSDM, WWW, CIKM, TALG, ESA, INFOSYS, SPE, CPM, DCC, ECIR, SPIRE, Algorithmica.}

%\section{Computer skills}
%\cvitem{OS}{Mac OS X, Linux.}
%\cvitem{Programming}{C++ and C (advanced); Java, Python.}
%\cvitem{Architecture}{Knowledge of modern CPUs architecture.}
%\cvitem{Typography}{\LaTeX, Pages.}
%\cvitem{Database}{MySQL, MongoDB.}

%\section{Professional Memberships}
%\cvitem{2019 -- present}{IEEE Italy Section.
%Membership number: 95683551.}
%
%\cvitem{2017 -- present}{Association for Computing Machinery (ACM SIGIR).
%Membership number: 6216262.}

\newpage

\section{Languages}
\cvitemwithcomment{Italian}{Native}{CEFR level: C2}
\cvitemwithcomment{English}{Fluent}{CEFR level: C1}
\cventry{2018}{TOEFL iBT in English}{}{}{}{100 (HIGH level)}
\cventry{2008}{First Certificate in English (Level B2)}{}{}{}{University of Cambridge, Cambridge, United Kingdom}

% \section{Driving Licences}
% \cvitem{01/10/2010}{Driving License of type B}
% \cvitem{23/01/2010}{Driving License of type A}

% \section{Web}
% \cvitem{Personal}{http://pages.di.unipi.it/pibiri}
% \cvitem{GitHub}{https://github.com/jermp}
% \cvitem{LinkedIn}{https://www.linkedin.com/in/giulio-ermanno-pibiri-496ab3ab/}

\end{document}