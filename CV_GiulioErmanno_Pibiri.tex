\documentclass[a4paper, sans]{moderncv}
\usepackage[utf8]{inputenc}

\moderncvstyle{casual}
\moderncvcolor{black}

\usepackage[left=2cm, bottom=3cm, right=2cm, top=5cm]{geometry}

\newcommand{\return}{\vspace{0.3cm}}

\newcommand{\red}[1]{{\color{red}{#1}}}

\firstname{Giulio Ermanno}
\familyname{Pibiri}

\title{Curriculum Vitae et Studiorum}

\begin{document}

\makecvtitle

\section{Contact Information}
\cvitem{}{``Ca' Foscari'' University of Venice}
\cvitem{}{Department of Environmental Sciences, Informatics and Statistics (DAIS)}
\cvitem{}{Via Torino 155, 30170 Mestre (Venice), Italy}
\cvitem{Email}{giulioermanno.pibiri@unive.it}
\cvitem{Personal Web page}{\url{https://jermp.github.io}}

\section{Personal Information}
\cvitem{Place of birth}{Bagno a Ripoli (Florence), Italy}
\cvitem{Date of birth}{13 July 1990}

\section{Education}

\cventry{01/11/2015 -- 31/10/2018}{PhD in Computer Science (INF/01)}{}{}{}
{\begin{itemize}
\item University of Pisa, Pisa, Italy
\item Thesis: \emph{Space- and Time-Efficient Data Structures for Massive Datasets}
\newline
(Defended on 08/03/2019)
\item Grade: Excellent
\item Supervisor: Rossano Venturini (\url{https://rossanoventurini.github.io})
\end{itemize}}

\cventry{2012 -- 2014}{Master Degree in Computer Science \& Networking (class LM18)}{}{}{}{
\begin{itemize}
\item University of Pisa and Scuola Superiore Sant'Anna, Pisa, Italy
\item Thesis: \emph{Dynamic Elias-Fano Encoding}
\newline
(Defended on 06/03/2015)
\item Grade: 110/110 \emph{summa cum laude}
\item Supervisor: Rossano Venturini (\url{https://rossanoventurini.github.io})
\end{itemize}}

\cventry{2009 -- 2012}{Bachelor Degree in Computer Engineering (class L08)}{}{}{}{
\begin{itemize}
\item University of Florence, Florence, Italy
\item Thesis: \emph{Quantum Computation \& Grover's Algorithm}
\newline
(Defended on 09/10/2012)
\item Grade: 110/110 \emph{summa cum laude}
\item Supervisor: Gabriele Vezzosi (\url{http://www.dma.unifi.it/~vezzosi})
\end{itemize}}

\cventry{2004 -- 2009}{High School Diploma}{}{}{}{
\begin{itemize}
\item Liceo Scientifico Statale Guido Castelnuovo, Florence, Italy
\item Grade: 100/100.
\end{itemize}
}



\section{Research Interests}

\cvitem{Keywords}{Data Structures, Data Compression, Indexing, Efficiency}

\cvitem{Short Description}{My research activity focuses on the design and implementation of compressed data structures for indexing large quantities of data coming from different fields of computing, such as Bioinformatics, Information Retrieval, and Natural Language Processing. The main objective is to improve the efficiency of complex tasks in these fields by providing efficient (i.e., \emph{fast}) and effective (i.e., \emph{small}) indexes to maintain/query data. In fact, a compressed index uses less storage space than the original data, thus permitting:
\begin{itemize}
\item for a fixed memory budget, to handle larger datasets;
\item for the same dataset, to maintain its compressed representation in faster memory levels (e.g., RAM instead of disk), hence granting faster access.
\end{itemize}
I am committed to efficient software production: my software is available on GitHub.}

\cvitem{Research Problems}{
\emph{Minimizer sampling schemes} (WABI 2024, AMB 2025);
\emph{Robustness Verification of Tree Ensembles} (CSS 2023, S$\&$P 2025);
\emph{Colored $k$-mer Indexing} (WABI 2023, RECOMB 2024, AMB 2024, JCB 2024);
\emph{Reference Indexing} (RECOMB 2023);
\emph{Locality-Preserving Minimal Perfect Hashing for $k$-mers} (BIOINF 2023);
\emph{Compressed and Weighted Dictionaries for $k$-mers} (BIOINF 2022, WABI 2022, AMB 2023, GBIO 2023);
\emph{Time Series Compression} (SPIRE 2021);
\emph{Minimal Perfect Hashing} (SIGIR 2021, TKDE 2023);
\emph{Prefix-Sums} (SPE 2020);
\emph{Rank/Select Queries} (INFOSYS 2021);
\emph{Query Auto-Completion} (SIGIR 2020);
\emph{Bitmap Compression} (DCC 2021);
\emph{Indexing of Semantic Relations} (TKDE 2020, ICDE 2021);
\emph{Indexing and Estimation of Language Models} (SIGIR 2017, TOIS 2019);
\emph{Inverted Index Compression} (TOIS 2017, EBDT 2018, WSDM 2019, TKDE 2019, CSUR 2020);
\emph{Succinct and Dynamic Ordered Sets of Integers} (CPM 2017).
}


\section{Research Positions}

\cventry{06/06/2022 -- present}{Assistant Professor of Computer Science (INF/01)}{}{}{}{
\begin{itemize}
\item Department of Environmental Sciences, Informatics and Statistics, ``Ca' Foscari'' University of Venice, Venice, Italy
\item Protocollo n. 52696, 01/06/2022
\item Scientific Habilitation for Associate Professorship for Computer Science (01/B1) and Computer Engineering (09/H1) earned
      on 19-th November 2024.
\end{itemize}
}

\cventry{15/03/2021 -- 31/05/2022}{Postdoctoral Research Fellow in Computer Science}{}{}{}{
\begin{itemize}
\item Institute of Science and Information Technologies ``A. Faedo'' (ISTI), National Research Council of Italy (CNR), Pisa, Italy
\item Research grant issued on the European project ACCORDION with theme ``Tecniche algoritmiche per compressione, indicizzazione e ricerca di grandi quantità di dati e progettazione di relative librerie software open source'' (Protocollo n. 0000901/2021, 09/03/2021, ISTI 004/2021 - PI)
% \item Reference publications: SIGIR 2021, SPIRE 2021, BIOINF 2022, WABI 2022
\end{itemize}
}

\cventry{01/11/2018 -- 28/02/2021}{Postdoctoral Research Fellow in Computer Science}{}{}{}{
\begin{itemize}
\item Institute of Science and Information Technologies ``A. Faedo'' (ISTI), National Research Council of Italy (CNR), Pisa, Italy
\item Research grant issued on the European project BIGDATAGRAPES with theme ``Compressione, indicizzazione e ricerca su grandi collezioni di dati semantici'' (Protocollo n. 0003847, 24/10/2018, ISTI 014/2018 - PI)
% \item Reference publications: TKDE 2019, TKDE 2020, SIGIR 2020, SPE 2020, INFOSYS 2021, DCC 2021, ICDE 2021
\end{itemize}
}

\cventry{01/11/2015 -- 31/10/2018}{PhD Student in Computer Science}{}{}{}{
\begin{itemize}
\item University of Pisa, Pisa, Italy
\item Thesis: \emph{Space- and Time-Efficient Data Structures for Massive Datasets}
\item Supervisor: Rossano Venturini (\url{https://rossanoventurini.github.io})
% \item Reference publications: TKDE 2019, TOIS 2019, WSDM 2019, SIGIR 2017, CPM 2017, TOIS 2017
    % \begin{itemize}
    %     \item ``On Optimally Partitioning Variable-Byte Codes'' (TKDE 2019) %, C++ code at \url{https://github.com/jermp/opt_vbyte};
    %     \item ``Handling Massive N-Gram Datasets Efficiently'' (TOIS 2019) %, C++ code at \url{https://github.com/jermp/tongrams};
    %     \item ``Fast Dictionary-based Compression for Inverted Indexes'' (WSDM 2019) %, C++ code at \url{https://github.com/jermp/dint};
    %     \item ``Inverted Index Compression'' (EBDT 2018)
    %     \item ``Efficient Data Structures for Massive N-Gram Datasets'' (SIGIR 2017) %, C++ code at \url{https://github.com/jermp/tongrams};
    %     \item ``Dynamic Elias-Fano Representation'' (CPM 2017)
    %     \item ``Clustered Elias-Fano Indexes'' (TOIS 2017) %, C++ code at \url{https://github.com/jermp/clustered_elias_fano_indexes}.
    % \end{itemize}
\end{itemize}
Part of the research was conducted \textbf{abroad} (6 months):
\begin{itemize}
    \item 01/05/2018 -- 01/10/2018
        \begin{itemize}
            \item The University of Melbourne, School of Computing and Information Systems, Melbourne, Australia
            \item Supervisor: Alistair Moffat (\url{https://people.eng.unimelb.edu.au/ammoffat})
            \item Worked on fast dictionary-based decoding of compressed inverted index data.
            % \item Reference publication: WSDM 2019
        \end{itemize}
    \item 01/04/2018 -- 30/04/2018
        \begin{itemize}
            \item RIKEN Advanced Intelligence Project (AIP), Tokyo, Japan
            \item Supervisor: Yasuo Tabei (\url{https://sites.google.com/site/yasuotabei})
            \item Worked on various problems, such as, string similarity search,
            trie indexing, rank/select indexes, and sparse matrix multiplication.
            % \item Reference publication: INFOSYS 2021
        \end{itemize}
\end{itemize}
}

\section{Research Visits}

\cvitem{19/10/2022 -- 21/10/2022}{
Inria, Rennes, France. Visiting dr. Pierre Peterlongo and dr. Karel Brinda.
}

\cvitem{10/05/2022 -- 12/05/2022}{
The University of Lille, CNRS, CRIStAL Laboratory, Lille, France. Visiting dr. Antoine Limasset and dr. Camille Marchet.
}

\cvitem{01/05/2018 -- 01/10/2018}{
The University of Melbourne, School of Computing and Information Systems, Melbourne, Australia. Visiting prof. Alistair Moffat and dr. Matthias Petri.
}

\cvitem{01/04/2018 -- 30/04/2018}{
RIKEN Advanced Intelligence Project (AIP), Tokyo, Japan. Visiting dr. Yasuo Tabei and dr. Shunsuke Kanda.
}



\section{Projects}

\cventry{01/09/2021 -- 31/05/2021}{Italian PON project OK-INSAID}{}{}{}{
\begin{itemize}
\item Institute of Science and Information Technologies ``A. Faedo'' (ISTI), National Research Council of Italy (CNR), Pisa, Italy
% \item Italian PON project OK-INSAID with theme ``Management, Compression and Indexing of Industrial Big Data''
\item Role: Investigator
\end{itemize}
}

\cventry{15/03/2021 -- 31/05/2021}{European project ACCORDION}{}{}{}{
\begin{itemize}
\item Institute of Science and Information Technologies ``A. Faedo'' (ISTI), National Research Council of Italy (CNR), Pisa, Italy
% \item European project ACCORDION with theme ``Tecniche algoritmiche per compressione, indicizzazione e ricerca di grandi quantità di dati e progettazione di relative librerie software open source''
\item Role: Investigator
\end{itemize}
}

\cventry{01/11/2018 -- 28/02/2021}{European project BIGDATAGRAPES}{}{}{}{
\begin{itemize}
\item Institute of Science and Information Technologies ``A. Faedo'' (ISTI), National Research Council of Italy (CNR), Pisa, Italy
% \item European project BIGDATAGRAPES with theme ``Compressione, indicizzazione e ricerca su grandi collezioni di dati semantici''
\item Role: \emph{Task Leader} for the task 3.3 ``Big Data Indexing''
% \item Description: The research activity conducted for this project focused on the design of time and space efficient indexing data structures for structured and unstructured data such as RDF graphs and text documents, including compression techniques for Big data management that support a broad range of analytical queries over arbitrary data dimensions.
% In particular, it resulted in the development of
% \begin{itemize}
%     \item a novel compressor for inverted indexes;
%     \item a novel compressed index for RDF data.
%     % \item a novel compressor for time series.
% \end{itemize}
% Both results have been published in IEEE Transactions on Knowledge and Data Engineering (TKDE),
% as the papers
% \begin{itemize}
% \item ``On Optimally Patitioning Variable-Byte Codes''
% \item ``Compressed Indexes for Fast Search of Semantic Data''
% \end{itemize}
% with the corresponding C++ libraries available on GitHub at
% \begin{itemize}
% \item \url{https://github.com/jermp/opt_vbyte}
% \item \url{https://github.com/jermp/rdf_indexes}
% \end{itemize}
% \item During this period, I also kept doing my own independent research on algorithms and data structures.
% The studied problems involved:
% inverted indexes (CSUR 2020),
% prefix-sums (SPE 2020),
% bitmap compression (DCC 2021),
% query auto-completion (SIGIR 2020),
% rank and select indexes for bitmaps (INFOSYS 2021),
% and minimal perfect hashing (SIGIR 2021).
% All works have been published in top-tier conferences/journals.
% The corresponding software libraries are available from my GitHub page.
\end{itemize}
}

\cventry{01/06/2017 -- 31/10/2018}{European project LIGA}{}{}{}{
\begin{itemize}
\item Institute of Science and Information Technologies ``A. Faedo'' (ISTI), National Research Council of Italy (CNR), Pisa, Italy
% \item European project \emph{Large-scale Indie Gaming Analytics} (LIGA)
\item Role: Principal Developer
% \item Description: The LIGA project aimed at designing and developing a proof-of-concept platform, customized for the 3D-KUMO use case (\url{https://www.3dkumo.com}), to analyze the huge volume volume of data generated by the users of indie games (i.e., players) on web portals and social networks.
% The purpose of the LIGA platform is to extend the engagement of indie games’ users beyond the gaming experience to the collection and sharing of game-related content, mainly related to 3D printing of game characters.
% To this end, the LIGA platform consists in three main components:
% (1) A High Performance Big Data Analysis (HPC-BDA) tool for the near-real time fetching and representation of online social network data related to indie games, including trends, players sentiment and network structure analysis;
% (2) A Smart Retail Recommendation Engine (REC-ENG) service integrated with the KUMO platform to suggests 3D models to customers, based on the information provided by the HPC-BDA tool;
% (3) A 3D Content Analytics Service and Dashboard (3D-DASH) service to provide content providers with insights and analysis on why contents are consumed and ``liked'', based on the information provided by the HPC-BDA tool.
\end{itemize}
}


% \section{Other Work Experience}

% \cventry{01/04/2015 -- 01/07/2015}{Software Engineer Intern}{}{}{}{
% \begin{itemize}
% \item IBM, Rome, Italy
% \item Supervisor: Alessio Fioravanti
% \item Worked on the design of the IBM Customer Partnership (Web) Portal for the management of IBM customers and projects.
% \end{itemize}
% }


% \section{Training Activity}

% \cventry{05/03/2016 -- 11/03/2016}{Bertinoro International Spring School (BISS)}{}{}{}{
% \begin{itemize}
% \item \emph{Advanced Topics in Programming Languages} (13 hours) -- Giuseppe Castagna, Université Paris Diderot – Paris 7
% \item \emph{Models and Languages for Service-Oriented and Cloud Computing} (13 hours) -- Gianluigi Zavattaro, University of Bologna
% \item \emph{Algorithmic methods for mining large graphs} (13 hours) -- Aristides Gionis, Aalto University
% \end{itemize}
% }


\section{Teaching Activity}

\cvitem{A.Y. 2024/2025}{Teacher for COMPUTER SCIENCE I - MOD. 1 (code CT0569, 6 CFU), Bachelor Degree in Ingegneria Fisica, Ca' Foscari University of Venice}

\cvitem{A.Y. 2024/2025}{Teacher for INTRODUCTION TO COMPUTER PROGRAMMING (code CT0665, 6 CFU), Bachelor Degree in Computer Science, Ca' Foscari University of Venice}

\cvitem{A.Y. 2024/2025}{Teacher for PROGRAMMING AND LABORATORY-2 (code CT0442, 6 CFU), Bachelor Degree in Computer Science, Ca' Foscari University of Venice}

\cvitem{A.Y. 2023/2024}{Teacher for AUTONOMOUS, DISTRIBUTED AND PERVASIVE SYSTEMS-2 (code PHD156-2, 2 CFU), PhD Program in Computer Science, Ca' Foscari University, Italy}

\cvitem{A.Y. 2023/2024}{Teacher for COMPUTER SCIENCE I - MOD. 1 (code CT0569, 6 CFU), Bachelor Degree in Ingegneria Fisica, Ca' Foscari University, Italy}

\cvitem{A.Y. 2023/2024}{Teacher for INTRODUCTION TO PROGRAMMING-1 (code CT0441, 6 CFU), Bachelor Degree in INFORMATICA, Ca' Foscari University, Italy}

\cvitem{A.Y. 2023/2024}{Teacher for PROGRAMMING AND LABORATORY-2 (code CT0442, 6 CFU), Bachelor Degree in INFORMATICA, Ca' Foscari University, Italy}

\cvitem{A.Y. 2022/2023}{Teacher for Introduction to Algorithms (3h), Invited lecture, IIS A. Pacinotti, Mestre (Venice), Italy}

\cvitem{A.Y. 2022/2023}{Teacher for KNOWLEDGE, INTERACTION AND INTELLIGENT SYSTEMS-2 (code PHD157-2, 2 CFU), PhD Program in Computer Science, Ca' Foscari University, Italy}

\cvitem{A.Y. 2022/2023}{Teacher for COMPUTER SCIENCE I - MOD. 1 (code CT0569, 6 CFU), Bachelor Degree in Ingegneria Fisica, Ca' Foscari University, Italy}

\cvitem{A.Y. 2022/2023}{Teacher for COMPUTER SCIENCE FOR CULTURAL HERITAGE (code CT0612, 6 CFU), Bachelor Degree in SCIENZE E TECNOLOGIE PER I BENI CULTURALI, Ca' Foscari University, Italy}

\cvitem{A.Y. 2022/2023}{Teacher for PROGRAMMING AND LABORATORY-2 (code CT0442, 6 CFU), Bachelor Degree in INFORMATICA, Ca' Foscari University, Italy}

% 11/04/2022 -- 15/04/2022
\cvitem{A.Y. 2021/2022}{Teacher for \emph{Theory and Practice of Data Compression} (5 CFU), PhD Program in Ingegneria dell'Informazione, University of Pisa, Italy}
% \cvitem{}{\url{https://github.com/jermp/data_compression_course}}

% 02/2020 -- 06/2020
\cvitem{A.Y. 2019/2020}{Teacher for \emph{Algorithmics and Laboratory - Corso B} (code 008AA, 3 CFU), Bachelor Degree in Computer Science, University of Pisa, Italy}
% \cvitem{}{\url{http://didawiki.cli.di.unipi.it/doku.php/informatica/all-b/start}}

% 02/2019 -- 06/2019
\cvitem{A.Y. 2018/2019}{Assistant for \emph{Algorithmics and Laboratory - Corso A} (code 008AA, 3 CFU), Bachelor Degree in Computer Science, University of Pisa, Italy}
% \cvitem{}{\url{http://didawiki.cli.di.unipi.it/doku.php/informatica/all-b/start}}

% 09/2018 -- 12/2018
\cvitem{A.Y. 2017/2018}{Assistant for \emph{Competitive Programming and Contests} (code 645AA, 6 hours), Master Degree in Computer Science, University of Pisa, Italy}
% \cvitem{}{\url{https://github.com/rossanoventurini/CompetitiveProgramming/tree/47111d935e5de6755890dd2951742bdf4857ce3b}}

% 09/2017 -- 12/2017
\cvitem{A.Y. 2016/2017}{Assistant for \emph{Competitive Programming and Contests} (code 645AA, 6 hours), Master Degree in Computer Science, University of Pisa, Italy}
% \cvitem{}{\url{https://github.com/rossanoventurini/CompetitiveProgramming/tree/9583aa75dc0e2e87c8b0a167d7978360ad9595f5}}

% 09/2016 -- 12/2016
\cvitem{A.Y. 2015/2016}{Teacher for \emph{Algorithmics and Laboratory - Corso di recupero} (code 008AA, 3 CFU), Bachelor Degree in Computer Science, University of Pisa, Italy}
% \cvitem{}{\url{http://didawiki.cli.di.unipi.it/doku.php/informatica/alr/start}}

% 02/2016 -- 06/2016
\cvitem{A.Y. 2015/2016}{Assistant for \emph{Algorithmics and Laboratory - Corso A} (code 008AA, 3 CFU), Bachelor Degree in Computer Science, University of Pisa, Italy}
% \cvitem{}{\url{http://didawiki.cli.di.unipi.it/doku.php/informatica/all-a/all16/start}}


\section{Awards and Grants}

\cvitem{2022}{\emph{Young Researcher Award} for the year 2021, issued by ISTI-CNR.}
% \cvitem{}{\url{https://www.isti.cnr.it/en/research/awards/young-researcher-award}}

\cvitem{2021}{\emph{Young Researcher Award} for the year 2020, issued by ISTI-CNR.}
% \cvitem{}{\url{https://www.isti.cnr.it/en/research/awards/young-researcher-award}}

\cvitem{2020}{\emph{Young Researcher Award} for the year 2019, issued by ISTI-CNR.}
% \cvitem{}{\url{https://www.isti.cnr.it/en/research/awards/young-researcher-award}}

\cvitem{2017}{\emph{SIGIR Student Travel Grant}, issued by ACM SIGIR.}
% \cvitem{}{\url{https://sigir.org/general-information/travel-grants}}

\cvitem{2015}{\emph{PhD Scholarship},
issued by the University of Pisa, Department of Computer Science.}
% \cvitem{}{\url{https://dottorato.di.unipi.it/phd-programme/people/alumni}}

\cvitem{2015}{\emph{Master Degree Award} for the A.Y. 2013/2014, issued by Scuola Superiore Sant'Anna.}
% \cvitem{}{\href{https://www.santannapisa.it/it/news/merito-nuovi-riconoscimenti-agli-studenti-della-laurea-magistrale-informatica-e-networking}{\texttt{https://www.santannapisa.it/it/news/merito}}}

\cvitem{2015}{\emph{Best Master Thesis Award in Theoretical Computer Science}, issued by the Italian chapter of the European Association for Theoretical Computer Science (EATCS).}
% \cvitem{}{\url{https://www.eatcs.org/index.php/italian-chapter}}



\section{Organizing Committees}

\cvitem{2025}{The 23-rd Symposium on Experimental and Algorithms (SEA).}
\cvitem{2023}{Co-Chair of the 18-th edition of the Workshop on Compression, Text, and Algorithms (WCTA), co-located with SPIRE 2023.}
\cvitem{2020}{The 28-th edition of the Annual European Symposium on Algorithms (ESA 2020).}
\cvitem{2019}{The 30-th edition of the International Symposium on Combinatorial Pattern Matching (CPM 2019).}
\cvitem{2017}{The 24-th International Symposium on String Processing and Information Retrieval (SPIRE 2017).}
\cvitem{2016}{The 39-th ACM International SIGIR Conference on Research and Development in Information Retrieval (SIGIR 2016). (As student volunteer.)}

\section{Program Committees}

\cvitem{2025}{The 32nd International Symposium on String Processing and Information Retrieval (SPIRE 2025).}
\cvitem{2025}{The 23-rd Symposium on Experimental and Algorithms (SEA 2025).}
\cvitem{2025}{The 15-th RECOMB Satellite Conference on Biological Sequence Analysis (RECOMB-SEQ 2025).}

\cvitem{2024}{The 47-th ACM International SIGIR Conference on Research and Development in Information Retrieval (SIGIR 2024).}
\cvitem{2024}{The 14-th RECOMB Satellite Conference on Biological Sequence Analysis (RECOMB-SEQ 2024).}

\cvitem{2023}{The 31-st European Symposium on Algorithms (ESA 2023 -- Track B).}
\cvitem{2023}{The 13-rd RECOMB Satellite Conference on Biological Sequence Analysis (RECOMB-SEQ 2023).}
\cvitem{2023}{The 46-th ACM International SIGIR Conference on Research and Development in Information Retrieval (SIGIR 2023).}
\cvitem{2023}{The 13-rd International Symposium on Algorithms and Complexity (CIAC 2023).}
\cvitem{2023}{The 45-th European Conference on Information Retrieval (ECIR 2023).}
\cvitem{2023}{The 16-th International ACM Conference on Web Search and Data Mining (WSDM 2023).}

\cvitem{2022}{The 45-th ACM International SIGIR Conference on Research and Development in Information Retrieval (SIGIR 2022).}
\cvitem{2022}{The 44-th European Conference on Information Retrieval (ECIR 2022).}
\cvitem{2022}{The 15-th International ACM Conference on Web Search and Data Mining (WSDM 2022).}

\cvitem{2021}{The 30-th ACM International Conference on Information and Knowledge Management (CIKM 2021).}
\cvitem{2021}{The 44-th ACM International SIGIR Conference on Research and Development in Information Retrieval (SIGIR 2021).}
\cvitem{2021}{The 43-rd European Conference on Information Retrieval (ECIR 2021).}
\cvitem{2021}{The 14-th International ACM Conference on Web Search and Data Mining (WSDM 2021).}

\cvitem{2020}{The 29-th ACM International Conference on Information and Knowledge Management (CIKM 2020).}
\cvitem{2020}{The 43-rd ACM International SIGIR Conference on Research and Development in Information Retrieval (SIGIR 2020).}

\cvitem{2019}{The 42-nd ACM International SIGIR Conference on Research and Development in Information Retrieval (SIGIR 2019).}

\section{Reviewing Activity}
\cvitem{2016 -- present}{I am or have been an anonymous reviewer for the following conferences/journals.}

\medskip
\subsection{\textbf{Conferences}}
\cvitem{}{
\begin{itemize}
    \item SIGIR -- ACM Conference on Research and Development in Information Retrieval
    \item WSDM -- ACM Conference on Web Search and Data Mining
    \item WWW -- The Web Conference
    \item CIKM -- ACM Conference on Information and Knowledge Management
    \item CPM -- Annual Symposium on Combinatorial Pattern Matching
    \item DCC -- IEEE Data Compression Conference
    \item ECIR -- European Conference on Information Retrieval
    \item ESA -- European Symposium on Algorithms
    \item SPIRE -- String Processing and Information Retrieval
    \item ISAAC -- International Symposium on Algorithms and Computation
    \item RECOMB -- International Conference on Research in Computational Molecular Biology
    \item RECOMB-SEQ -- Satellite Conference on Biological Sequence Analysis
    \item CIAC -- International Symposium on Algorithms and Complexity
    \item SODA -- Symposium on Discrete Algorithms
\end{itemize}
}

\subsection{\textbf{Journals}}
\cvitem{}{
\begin{itemize}
    \item ACM Transactions on Algorithms
    \item Bioinformatics
    \item Genome Biology
    \item Information Systems
    \item Software: Practice and Experience
    \item Journal of Experimental Algorithmics
    \item Algorithmica
    \item MDPI Algorithms
    \item IEEE Transactions on Information Forensics and Security
    \item Information Processing and Management
    \item PCI Math \& Comp. Biol.
\end{itemize}
}

\section{Editorial Service}

\cvitem{2024 -- present}{I am serving as a guest editor for the Special Issue ``Computation over compressed data'' (selected papers
from DCC 2024) in Information Systems.}

\cvitem{2024 -- present}{I am member of the editorial board of \href{https://www.sciencedirect.com/journal/information-systems/about/aims-and-scope}{\emph{Information Systems}}.}{
% https://www.sciencedirect.com/journal/information-systems/about/editorial-board
% https://doi.org/10.1016/S0306-4379(24)00031-0
}

\cvitem{2024 -- present}{I serve as a recommender (similar to an editor)
of PCI \href{https://mcb.peercommunityin.org}{\emph{Mathematical and Computational Biology}}.}{
% https://mcb.peercommunityin.org/about/recommenders
}

\section{Talks}

\medskip
\subsection{\textbf{Invited Keynote}}

\cvitem{22/11/2023}{\emph{Compressing and indexing pangenomes with meta-colored compacted de Bruijn graphs}. ALPACA-PANGAIA Annual Workshop. Amsterdam, Netherlands.}
\cvitem{20/10/2022}{\emph{Modular reference indexing with the de Bruijn graph: overview and challenges}. Workshop on Indexing Omic Sequences. Rennes, France.}
\cvitem{10/05/2022}{\emph{On Weighted K-Mer Dictionaries}. TUDASTIC 2022 (TUtorials on DAta Structures for Text Indexation and Compression). Lille, France.}


\medskip
\subsection{\textbf{Conferences}}

\cvitem{02/09/2024}{\emph{The Mod-Minimizer: A Simple and Efficient Sampling Algorithm for Long K-Mers}. WABI 2024. Egham, UK.}
\cvitem{30/04/2024}{\emph{Meta-Colored Compacted de Bruijn Graphs}. RECOMB 2024, Boston, USA.}
\cvitem{25/07/2023}{\emph{Locality-Preserving Minimal Perfect Hashing of K-Mers}. ISMB 2023, Lyon, France.}
\cvitem{21/03/2023}{\emph{Spectrum Preserving Tilings Enable Sparse and Modular Reference Indexing}. DSB 2023, Delft, Netherlands.}
\cvitem{05/09/2022}{\emph{On Weighted K-Mer Dictionaries}. WABI 2022, Potsdam, Germany.}
\cvitem{13/07/2022}{\emph{Sparse and Skew Hashing of K-Mers}. ISMB 2022. Madison, Wisconsin, USA.}
\cvitem{14/06/2022}{\emph{PTHash: Revisiting FCH Minimal Perfect Hashing}. DSB 2022, D\"usseldorf, Germany.}
\cvitem{20/05/2022}{\emph{Sparse and Skew Hashing of K-Mers}. RECOMB-seq 2022. La Jolla, California, USA.}
\cvitem{04/10/2021}{\emph{TSXor: A Simple Time-Series Compression Algorithm}. SPIRE 2021. Lille, France (Virtual event).}
\cvitem{14/09/2021}{\emph{PTHash: Revisiting FCH Minimal Perfect Hashing}. IIR 2021. Polytechnic University of Bari, Bari (Virtual event).}
\cvitem{07/2021}{\emph{PTHash: Revisiting FCH Minimal Perfect Hashing}. SIGIR 2021. Montreal, Canada (Virtual event).}
\cvitem{04/2021}{\emph{Compressed Indexes for Fast Search of Semantic Data}. ICDE 2021. Chania, Greece (Virtual event).}
\cvitem{03/2021}{\emph{Fast and Compact Set Intersection through Recursive Universe Partitioning}. DCC 2021. Snow Bird, USA (Virtual event).}
\cvitem{27/07/2020}{\emph{Efficient and Effective Query Auto-Completion}. SIGIR 2020. Xi'An, China (Virtual event).}
\cvitem{17/09/2019}{\emph{Compressed Indexes for Fast Search of Semantic Data}. IIR 2019. Department of Information Engineering, Padova, Italy.}
\cvitem{12/02/2019}{\emph{Fast Dictionary-based Compression for Inverted Indexes}. WSDM 2019. Melbourne Exhibition Center, Melbourne, Australia.}
\cvitem{10/08/2017}{\emph{Efficient Data Structures for Massive N-Gram Datasets}. SIGIR 2017. Keio Plaza Hotel, Tokyo, Japan.}
\cvitem{06/07/2017}{\emph{Dynamic Elias-Fano Representation}. CPM 2017. University Library of Warsaw, Warsaw, Poland.}
\cvitem{06/06/2017}{\emph{Efficient Data Structures for Massive N-Gram Datasets}. IIR 2017. Universit\`{a} della Svizzera Italiana, Lugano, Switzerland.}


\medskip
\subsection{\textbf{Seminars}}

\cvitem{02/11/2023}{\emph{Compressing and indexing pangenomes with meta-colored compacted de Bruijn graphs}. ISTI-CNR, Pisa, Italy (Virtual event).}
\cvitem{19/05/2022}{\emph{Sparse and Skew Hashing of K-Mers}. ISTI-CNR, Pisa, Italy (Virtual event).}
\cvitem{22/12/2021}{\emph{Minimal Perfect Hashing and K-Mer String Dictionaries}. ``Ca' Foscari'' University of Venice, Venice, Italy (Virtual event).}
\cvitem{16/11/2021}{\emph{PTHash: Revisiting FCH Minimal Perfect Hashing}. ISTI-CNR, Pisa, Italy (Virtual event).}
\cvitem{04/03/2021}{\emph{Efficiency for Real-World Applications}. ISTI-CNR, Pisa, Italy (Virtual event).}
\cvitem{07/06/2019}{\emph{Ordered Set Problems}. ISTI-CNR, Pisa, Italy.}
\cvitem{01/02/2019}{\emph{Indexing Compressed Data for Fast Retrieval}. University of Pisa, Pisa, Italy.}
\cvitem{29/10/2018}{\emph{Effective Web Graph Representations}. University of Pisa, Pisa, Italy.}
\cvitem{17/05/2018}{\emph{On Optimally Partitioning Variable-Byte Index Data}. RMIT University, Melbourne, Australia.}
\cvitem{10/04/2018}{\emph{Elias-Fano Encoding: a powerful tool for data structure design}.  RIKEN AIP, Tokyo, Japan.}
\cvitem{21/06/2016}{\emph{Elias-Fano Encoding: succinct representation of monotone integer sequences with search operations}. University of Pisa, Pisa, Italy.}





\section{Publications}

\bigskip
\subsection{\textbf{Journal Papers}}

\cvitem{AMB 2025}{Ragnar Groot Koerkamp and Daniel Liu and Giulio Ermanno Pibiri. \emph{The open-closed mod-minimizer algorithm}. 2025. Algorithms for Molecular Biology (AMB), pages 17.
\newline
DOI: \href{https://almob.biomedcentral.com/articles/10.1186/s13015-025-00270-0}{10.1186/s13015-025-00270-0}
\newline
ISSN: 1748-7188}

\cvitem{JCB 2024}{Alessio Campanelli and Giulio Ermanno Pibiri and Jason Fan and Rob Patro. \emph{Where the patterns are: repetition-aware compression for colored de Bruijn graphs}. 2024. Journal of Computational Biology (JCB), pages 29.
\newline
DOI: \href{https://doi.org/10.1089/cmb.2024.0714}{10.1089/cmb.2024.0714}
\newline
ISSN: 1557-8666}

\cvitem{BIOINF 2024}{Moein Karami and Aryan Soltani Mohammadi and Marcel Martin and Barış Ekim and Wei Shen and  Lidong Guo and Mengyang Xu and Giulio Ermanno Pibiri and Rob Patro and Kristoffer Sahlin. \emph{Designing efficient randstrobes for sequence similarity analyses}. 2024. Bioinformatics (BIOINF), pages 9.
\newline
DOI: \href{https://doi.org/10.1093/bioinformatics/btae187}{10.1093/bioinformatics/btae187}
\newline
ISSN: 1367-4811}

\cvitem{AMB 2024}{Jason Fan, Noor Pratap Singh, Jamshed Khan, Giulio Ermanno Pibiri, Rob Patro. \emph{Fulgor: A fast and compact $k$-mer index for large-scale matching and color queries}. 2024. Algorithms for Molecular Biology (AMB), pages 21.
\newline
DOI: \href{https://doi.org/10.1186/s13015-024-00251-9}{10.1186/s13015-024-00251-9}
\newline
ISSN: 1748-7188}

\cvitem{TKDE 2023}{Giulio Ermanno Pibiri and Roberto Trani. \emph{Parallel and External-Memory Construction of Minimal Perfect Hash Functions with PTHash}. 2023. IEEE Transactions on Knowledge and Data Engineering (TKDE), pages 12.
\newline
DOI: \href{https://ieeexplore.ieee.org/document/10210677}{10.1109/TKDE.2023.3303341}
\newline
ISSN: 1041-4347}

\cvitem{GBIO 2023}{Sebastian Schmidt, Shahbaz Khan, Jarno Alanko, Giulio Ermanno Pibiri, Alexandru I. Tomescu. \emph{Matchtigs: minimum plain text representation of $k$-mer sets}. 2023. Genome Biology (GBIO), pages 33.
\newline
DOI: \href{https://genomebiology.biomedcentral.com/articles/10.1186/s13059-023-02968-z}{10.1186/s13059-023-02968-z}
\newline
ISSN: 1474-7596}

\cvitem{BIOINF 2023}{Giulio Ermanno Pibiri, Yoshihiro Shibuya, and Antoine Limasset. \emph{Locality-Preserving Minimal Perfect Hashing of $k$-mers}. 2023. Bioinformatics (BIOINF), pages 9.
\newline
DOI: \href{https://academic.oup.com/bioinformatics/article/39/Supplement_1/i534/7210438}{10.1093/bioinformatics/btad219}
\newline
ISSN: 1367-4803}

\cvitem{AMB 2023}{Giulio Ermanno Pibiri. \emph{On Weighted $k$-mer Dictionaries}. 2023. Algorithms for Molecular Biology (AMB), pages 25.
\newline
DOI: \href{https://almob.biomedcentral.com/articles/10.1186/s13015-023-00226-2}{10.1186/s13015-023-00226-2}
\newline
ISSN: 1748-7188}

\cvitem{BIOINF 2022}{Giulio Ermanno Pibiri. \emph{Sparse and Skew Hashing of $k$-mers}. 2022. Bioinformatics (BIOINF), pages 9.
\newline
DOI: \href{https://doi.org/10.1093/bioinformatics/btac245}{10.1093/bioinformatics/btac245}
\newline
ISSN: 1367-4803}

\cvitem{INFOSYS 2021}{Giulio Ermanno Pibiri and Shunsuke Kanda. \emph{Rank/Select Queries over Mutable Bitmaps}. 2021. Information Systems (INFOSYS), pages 15.
\newline
DOI: \href{https://www.sciencedirect.com/science/article/pii/S0306437921000235}{10.1016/j.is.2021.101756}
\newline
ISSN: 0306-4379}

\cvitem{CSUR 2020}{Giulio Ermanno Pibiri and Rossano Venturini. \emph{Techniques for Inverted Index Compression}. 2020. ACM Computing Surveys (CSUR), pages 36.
\newline
DOI: \href{https://dl.acm.org/doi/10.1145/3415148}{10.1145/3415148}
\newline
ISSN: 0360-0300}

\cvitem{SPE 2020}{Giulio Ermanno Pibiri and Rossano Venturini. \emph{Practical Trade-Offs for the Prefix-Sum Problem}. 2020. Software: Practice and Experience (SPE), pages 29.
\newline
DOI: \href{https://onlinelibrary.wiley.com/doi/10.1002/spe.2918}{10.1002/spe.2918}
\newline
ISSN: 0038-0644}

\cvitem{TKDE 2020}{Raffaele Perego, Giulio Ermanno Pibiri and Rossano Venturini. \emph{Compressed Indexes for Fast Search of Semantic Data}. 2020. IEEE Transactions on Knowledge and Data Engineering (TKDE), pages 12.
\newline
DOI: \href{https://ieeexplore.ieee.org/document/8959165}{10.1109/TKDE.2020.2966609}
\newline
ISSN: 1041-4347}

\cvitem{TKDE 2019}{Giulio Ermanno Pibiri and Rossano Venturini. \emph{On Optimally Partitioning Variable-Byte Codes}. 2019. IEEE Transactions on Knowledge and Data Engineering (TKDE), pages 12.
\newline
DOI: \href{https://ieeexplore.ieee.org/document/8691421}{10.1109/TKDE.2019.2911288}
\newline
ISSN: 1041-4347}

\cvitem{TOIS 2019}{Giulio Ermanno Pibiri and Rossano Venturini. \emph{Handling Massive N-Gram Datasets Efficiently}. 2019. ACM Transactions on Information Systems (TOIS), pages 41.
\newline
DOI: \href{https://dl.acm.org/doi/10.1145/3302913}{10.1145/3302913}
\newline
ISSN: 1046-8188}

\cvitem{TOIS 2017}{Giulio Ermanno Pibiri and Rossano Venturini. \emph{Clustered Elias-Fano Indexes}. 2017. ACM Transactions on Information Systems (TOIS), volume 2, pages 33.
\newline
DOI: \href{https://dl.acm.org/doi/10.1145/3052773}{10.1145/3052773}
\newline
ISSN: 1046-8188}



\bigskip
\subsection{\textbf{Conference Papers}}

\cvitem{SEA 2025}{Lorraine A. K. Ayad and Gabriele Fici and Ragnar Groot Koerkamp and Grigorios Loukides and Rob Patro and Giulio Ermanno Pibiri and Solon P. Pissis. \emph{U-index: A Universal Indexing Framework for Matching Long Patterns}. 2025.
The 23rd Symposium on Experimental Algorithm (SEA), 18 pages.
\newline
DOI: red{\href{XXX}{XXX}}
\newline
ISBN: \red{XXX}
}

\cvitem{S$\&P$ 2025}{Stefano Calzavara, Lorenzo Cazzaro, Claudio Lucchese, and Giulio Ermanno Pibiri. \emph{Verifiable Boosted Tree Ensembles}. 2025.
The 46th IEEE Symposium on Security and Privacy (S$\&P$), 15 pages.
\newline
DOI: \href{https://www.computer.org/csdl/proceedings-article/sp/2025/223600a022/21B7QdnSnoQ}{10.1109/SP61157.2025.00022}
\newline
ISBN: 9798331522360
}

\cvitem{WABI 2024}{Ragnar Groot Koerkamp and Giulio Ermanno Pibiri. \emph{The mod-minimizer: a simple and efficient sampling algorithm for long $k$-mers}. 2024.
International Workshop on Algorithms in Bioinformatics (WABI), 22 pages.
\newline
DOI: \href{https://drops.dagstuhl.de/entities/document/10.4230/LIPIcs.WABI.2024.11}{10.4230/LIPIcs.WABI.2024.11}
\newline
ISBN: 9783959773409
}

\cvitem{ESA 2024}{Hermann, Stefan, Hans-Peter Lehmann, Giulio Ermanno Pibiri, Peter Sanders, and Stefan Walzer. \emph{PHOBIC: Perfect Hashing with Optimized Bucket Sizes and Interleaved Coding}. 2024.
European Symposium on Algorithms (ESA), 17 pages.
\newline
DOI: \href{https://drops.dagstuhl.de/entities/document/10.4230/LIPIcs.ESA.2024.69}{10.4230/LIPIcs.ESA.2024.69}
\newline
ISBN: 9783959773386
}

\cvitem{RECOMB 2024}{Giulio Ermanno Pibiri, Jason Fan, and Rob Patro. \emph{Meta-colored compacted de Bruijn graphs}. 2024.
International Conference on Research in Computational Molecular Biology (RECOMB), 16 pages.
\newline
DOI: \href{https://doi.org/10.1007/978-1-0716-3989-4_9}{10.1007/978-1-0716-3989-4\_9}
\newline
ISBN: 9781071639894
}

\cvitem{CCS 2023}{Stefano Calzavara, Lorenzo Cazzaro, Giulio Ermanno Pibiri, and Nicola Prezza. \emph{Verifiable Learning for Robust Tree Ensembles}. 2023.
ACM SIGSAC Conference on Computer and Communications Security (CCS), 16 pages.
\newline
DOI: \href{https://doi.org/10.1145/3576915.3623100}{10.1145/3576915.3623100}
\newline
ISBN: 9781450394505
}

\cvitem{WABI 2023}{Jason Fan, Noor Pratap Singh, Jamshed Khan, Giulio Ermanno Pibiri, and Rob Patro. \emph{Fulgor: A fast and compact $k$-mer index for large-scale matching and color queries}. 2023.
International Workshop on Algorithms in Bioinformatics (WABI), 21 pages.
\newline
DOI: \href{https://drops.dagstuhl.de/opus/volltexte/2023/18644/}{10.4230/LIPIcs.WABI.2023.18}
\newline
ISBN: 9783959772433
}

\cvitem{RECOMB 2023}{Jason Fan, Jamshed Khan, Giulio Ermanno Pibiri, and Rob Patro. \emph{Spectrum preserving tilings enable sparse and modular reference indexing}. 2023.
International Conference on Research in Computational Molecular Biology (RECOMB), 20 pages.
\newline
DOI: \href{https://link.springer.com/chapter/10.1007/978-3-031-29119-7_2}{10.1007/978-3-031-29119-7\_2}
\newline
ISBN: 9783031291197
}

\cvitem{WABI 2022}{Giulio Ermanno Pibiri. \emph{On Weighted $k$-mer Dictionaries}. 2022.
International Workshop on Algorithms in Bioinformatics (WABI), 20 pages.
\newline
DOI: \href{https://drops.dagstuhl.de/opus/volltexte/2022/17043}{10.4230/LIPIcs.WABI.2022.9}
\newline
ISBN: 9783959772433
}

\cvitem{SPIRE 2021}{Andrea Bruno, Franco Maria Nardini, Giulio Ermanno Pibiri, Roberto Trani, and Rossano Venturini. \emph{TSXor: A Simple Time Series Compression Algorithm}. 2021.
International Symposium on String Processing and Information Retrieval (SPIRE),
8 pages.
\newline
DOI: \href{https://doi.org/10.1007/978-3-030-86692-1_18}{10.1007/978-3-030-86692-1\_18}
\newline
ISBN: 9783030866921
}

\cvitem{SIGIR 2021}{Giulio Ermanno Pibiri and Roberto Trani. \emph{PTHash: Revisiting FCH Minimal Perfect Hashing}. 2021. ACM Conference on Research and Development in Information Retrieval (SIGIR), pages 10.
\newline
DOI: \href{https://dl.acm.org/doi/10.1145/3404835.3462849}{10.1145/3404835.3462849}
\newline
ISBN: 9781450380379}

\cvitem{DCC 2021}{Giulio Ermanno Pibiri. \emph{Fast and Compact Set Intersection through Recursive Universe Partitioning}. 2021. IEEE Data Compression Conference (DCC), pages 10.
\newline
DOI: \href{https://ieeexplore.ieee.org/document/9418701}{10.1109/DCC50243.2021.00037}
\newline
ISBN: 9781665403337}

\cvitem{SIGIR 2020}{Simon Gog, Giulio Ermanno Pibiri and Rossano Venturini. \emph{Efficient and Effective Query Auto-Completion}. 2020. ACM Conference on Research and Development in Information Retrieval (SIGIR), pages 10.
\newline
DOI: \href{https://dl.acm.org/doi/10.1145/3397271.3401432}{10.1145/3397271.3401432}
\newline
ISBN: 9781450380164}

\cvitem{WSDM 2019}{Giulio Ermanno Pibiri, Matthias Petri, and Alistair Moffat. \emph{Fast Dictionary-based Compression for Inverted Indexes}. 2019. ACM Conference on Web Search and Data Mining (WSDM), pages 9.
\newline
DOI: \href{https://dl.acm.org/doi/10.1145/3289600.3290962}{10.1145/3289600.3290962}
\newline
ISBN: 9781450359405}

\cvitem{SIGIR 2017}{Giulio Ermanno Pibiri and Rossano Venturini. \emph{Efficient Data Structures for Massive N-Gram Datasets}. 2017. ACM Conference on Research and Development in Information Retrieval (SIGIR), pages 10.
\newline
DOI: \href{https://dl.acm.org/doi/10.1145/3077136.3080798}{10.1145/3077136.3080798}
\newline
ISBN: 9781450350228}

\cvitem{CPM 2017}{Giulio Ermanno Pibiri and Rossano Venturini. \emph{Dynamic Elias-Fano Representation}. 2017. Annual Symposium on Combinatorial Pattern Matching (CPM), pages 14.
\newline
DOI: \href{https://drops.dagstuhl.de/opus/volltexte/2017/7324}{10.4230/LIPIcs.CPM.2017.30}
\newline
ISBN: 9783959770392}



\bigskip
\subsection{\textbf{Posters}}

\cvitem{ICDE 2021}{Raffaele Perego, Giulio Ermanno Pibiri and Rossano Venturini. \emph{Compressed Indexes for Fast Search of Semantic Data}. 2021. IEEE International Conference on Data Engineering (ICDE), pages 2.
\newline
DOI: \href{https://ieeexplore.ieee.org/document/9458814}{10.1109/ICDE51399.2021.00248}
\newline
ISBN: 9781728191850}


\bigskip
\subsection{\textbf{PhD Thesis}}

\cvitem{2019}{Giulio Ermanno Pibiri. \emph{Space- and Time-Efficient Data Structures for Massive Datasets}. 2019. Ph.D. Thesis, University of Pisa, 210 pages.}


\bigskip
\subsection{\textbf{Chapters}}

\cvitem{EBDT 2018}{Giulio Ermanno Pibiri and Rossano Venturini. \emph{Inverted Index Compression}. 2018. Encyclopedia of Big Data Technologies (EBDT), pages 8.
\newline
DOI: \href{https://link.springer.com/referenceworkentry/10.1007\%2F978-3-319-63962-8\_52-1}{10.1007/978-3-319-63962-8\_52-1}
\newline
ISBN: 9783319639628}


\section{Software}

\cvitem{GitHub profile}{
All software is open-source and available from \url{https://github.com/jermp}.
}

\bigskip
\subsection{\textbf{Data Structures}}

\cvitem{}{Efficient C++ implementations of the following data structures (see also related publications):
\begin{itemize}
\item Inverted Indexes (TOIS 2017, TKDE 2019, WSDM 2019, SIGIR 2020, CSUR 2020)
\item Tries (SIGIR 2017, TOIS 2019, TKDE 2020)
\item Compressed Bitmaps (DCC 2021)
\item Mutable Bitmaps with Rank/Select (INFOSYS 2021)
\item Segment-Trees and Fenwick-Trees (SPE 2020)
\item Minimal Perfect Hash Functions (SIGIR 2021, TKDE 2023, ESA 2024), Locality-Preserving Minimal Perfect Hash Functions (BIOINF 2023)
\item Dictionaries for $k$-mers (BIOINF 2022, WABI 2022, AMB 2023, GBIO 2023)
\item Colored $k$-mer indexes (WABI 2023, AMB 2024, RECOMB 2024, JCB 2024)
\end{itemize}
A more detailed list follows below.
}

\cvitem{\href{https://github.com/jermp/minimizers}{\textbf{minimizers}}}{
A collection of minimizer-based sampling algorithms.
\newline
Reference publication: WABI 2024, AMB 2025.
}

\cvitem{\href{https://github.com/jermp/fulgor}{\textbf{fulgor}}}{
Fulgor: A fast and compact $k$-mer index for large-scale matching and color queries.
\newline
Reference publication: RECOMB 2023, WABI 2023, AMB 2024, RECOMB 2024, JCB 2024.
}

\cvitem{\href{https://github.com/jermp/sshash}{\textbf{sshash}}}{
SSHash: A compressed, weighted, associative, exact dictionary for k-mers.
(See also the membership-only version \href{https://github.com/jermp/sshash-lite}{\textbf{sshash-lite}}.)
\newline
Reference publication: BIOINF 2022, WABI 2022, AMB 2023, GBIO 2023.
}

\cvitem{\href{https://github.com/jermp/lphash}{\textbf{lphash}}}{
LPHash: Fast and compact locality-preserving minimal perfect hashing for $k$-mer sets.
\newline
Reference publication: BIOINF 2023.
}


\cvitem{\href{https://github.com/jermp/pthash}{\textbf{pthash}}}{
PTHash: Fast and compact minimal perfect hash functions.
\newline
Reference publications: SIGIR 2021, TKDE 2023, ESA 2024.
}

\cvitem{\href{https://github.com/jermp/mutable_rank_select}{\textbf{rank\_select}}}{
Mutable bitmaps with support for Rank and Select queries.
\newline
Reference publication: INFOSYS 2021.
}

\cvitem{\href{https://github.com/jermp/psds}{\textbf{psds}}}{
A range of tree-shaped data structures for maintaining prefix-sums, including:
\begin{itemize}
\item binary Segment-Tree (top-down and bottom-up),
\item b-ary Segment-Tree,
\item Fenwick-Tree,
\item b-ary Fenwick-Tree,
\item blocked Fenwick-Tree,
\item truncated Fenwick-Tree.
\end{itemize}
Reference publication: SPE 2020.
}

\cvitem{\href{https://github.com/jermp/autocomplete}{\textbf{autocomplete}}}{Efficienct and effective
autocompletion framework, based on forward/inverted indexes, succinct RMQ, and string dictionaries (Front-Coding and tries).
\newline
Reference publication: SIGIR 2020.
}

\cvitem{\href{https://github.com/jermp/2i_bench}{\textbf{2i\_bench}}}{
A benchmarking suite for inverted index data structures, featuring the following compressors:
\begin{itemize}
\item Elias-Fano and partitioned Elias-Fano,
\item Opt-PFor-Delta,
\item Binary Interpolative,
\item QMX,
\item Simple family,
\item Variable-Byte family, including Opt-VByte,
\item Gamma, Delta, Rice, Zeta,
\item DINT.
\end{itemize}
Reference publication: CSUR 2020.
}

\cvitem{\href{https://github.com/jermp/interpolative_coding}{\textbf{interp}}}{
An efficient implementation of the Binary Interpolative Coding
algorithm.
}

\cvitem{\href{https://github.com/jermp/s_indexes}{\textbf{s\_indexes}}}{
Compressed bitmap indexes that support fast intersection and union.
\newline
Reference publication: DCC 2021.
}

\cvitem{\href{https://github.com/jermp/rdf_indexes}{\textbf{rdf\_indexes}}}{
Trie-based indexes for semantic data like RDF triples.
\newline
Reference publication: TKDE 2020.
}

\cvitem{\href{https://github.com/jermp/dint}{\textbf{dint}}}{
DINT: fast and compact dictionary-based decoder for inverted lists.
\newline
Reference publication: WSDM 2019.
}

\cvitem{\href{https://github.com/jermp/opt_vbyte}{\textbf{opt\_vbyte}}}{
Optimal partitioning of inverted lists compressed using binary vectors
and point-wise encoders, like Variable-Byte.
\newline
Reference publication: TKDE 2019.
}

\cvitem{\href{https://github.com/jermp/tongrams}{\textbf{tongrams}}}{
Fast language model queries and estimation in compressed space.
\newline
Reference publications: SIGIR 2017, TOIS 2019.
}

\cvitem{\href{https://github.com/jermp/clustered_elias_fano_indexes}{\textbf{clustered\_indexes}}}{
Clustered Elias-Fano inverted indexes.
\newline
Reference publication: TOIS 2017.
}

\bigskip
\subsection{\textbf{Miscellanea}}

\cvitem{\href{https://github.com/jermp/essentials}{\textbf{essentials}}}{A \textsf{C++} library providing essential core utilities for data structure design and benchmarking.
More precisely:
\begin{itemize}
\item benchmarking facilities, including: messages displaying local time, configurable timer class, function to prevent code elision by compiler, simple creation and printing of json documents;
\item functions to serialize-to and load-from disk data structures,
\item functions to compute the number of bytes consumed by data structures,
\item support for creating, removing, and iterate inside directories,
\item transparent support for contiguous memory allocation.
\end{itemize}
}

\cvitem{\href{https://github.com/jermp/cmd_line_parser}{\textbf{cmd\_line}}}{
Command line parser for C++17. It offers all handy features in just 150 lines of code.
}

\cvitem{\href{https://github.com/jermp/mm_file}{\textbf{mm\_file}}}{
A self-contained, header-only, implementation of memory-mapped files in C++ for both reading and writing.
}

\newpage

\section{Languages}
\cvitemwithcomment{Italian}{Native}{CEFR level: C2}
\cvitemwithcomment{English}{Fluent}{CEFR level: C1}
\cventry{2018}{TOEFL iBT in English}{}{}{}{100 (HIGH level)}
\cventry{2008}{First Certificate in English (Level B2)}{}{}{}{University of Cambridge, Cambridge, United Kingdom}

\end{document}
